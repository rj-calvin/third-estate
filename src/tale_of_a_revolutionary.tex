\startcomponent tale_of_a_revolutionary

\scene

\open \eliza is at a desk with the ten epochs of the tableau laid out before
her.

\starteliza
One, two, three, four, five, six, seven, eight, nine... ten epochs of human
history~- my, my...
\stop

\stage \eliza begins with the first epoch...

\stage She then changes her mind, she'll instead begin by reading the end.

\stage Pause. \eliza reads from the start of the last epoch.

\stage She then changes her mind, she instead wishes to see the last page of
the last epoch.

\stage She takes this page and after a moment recognizes that this is worth
reading:

\starteliza
How consoling for the philosopher who laments the errors, the crimes, the
injustices~- which still pollute the Earth~- is this view of the human race:

emancipated from its shackles;

released from the empire of chance, built by the enemies of its progress;

advancing with a firm and sure step along the path of truth, virtue, and
happiness.

\beat

It is the contemplation of this world that rewards him for all his efforts to
assist the progress of reason and the defense of liberty.

He dares to link these strivings to the eternal chain of human destiny, and in
this persuasion, he learns the true reward of virtue:

the pleasure of having done some lasting good which fate can no longer destroy
by a sinister stroke of revenge.

\beat

Such contemplation is for him a refuge where the memory of his persecutors can
no longer pursue him.

There he lives in thought, with men and women restored to their natural rights
and dignity, where the suffering and corruption of greed, envy, or fear has
long been forgotten.

\beat

There he truly lives with his peers in the Elysium that his reason has created
for him, and that his love for humanity enhances with the purest joys.
\stop

\stage Pause. Silence.

\stage \eliza takes in a deep breath before~--

\stopcomponent
