\startcomponent tale_of_a_philosopher

\scene

\open The \vernet residence living room.

\stage Enter \vernet.

\stage She sets the table with a breakfast of a twelve-egg omelette and
coffee.

\stage Enter \eliza.

\startvernet
Good evening. I've prepared you breakfast.
\stop

\starteliza
It's near daybreak.
\stop

\startvernet
By the look in your eyes, dear, it seems like you're still living in last
night.

\beat

Oh! What am I saying.

\beat{embraces \eliza}

I shouldn't say such things about a woman's appearance.

I apologize, darling. I made you breakfast.
\stop

\starteliza
This omelette is enormous. How many eggs did you put in it?
\stop

\startvernet
Twelve.
\stop

\starteliza
What? Twelve?

\beat{chuckles}

My attention must be worth a fortune!
\stop

\startvernet
It is to me, dear. Even if through bagged eyes.
\stop

\stage \eliza begins to eat.

\startvernet
Make yourself comfortable, dear, for I shall now share with you a tale of
despair!

A tale of hope!

A tale of the evil's of man and the power of true love!

\beat{with flourish}

AH! Our story begins, darling!

\beat

It was a stormy, summer day~- not of an unlike nature to the winds we hear
against these very walls.

When against this very door was the knock that had forever shaped my fate and
my role in yours.

\beat{performative}

I open the door to be presented with a soaked physician - your uncle, \cabanis.

His breath was heavy, as if he had just ran across the face of Paris in a hurry
to enjoy the sight of my own.

\beat

And he says to me, "I am a friend of Pinel and Boyer, do you have a room
available for rent?"

And I watch as the man patiently waited for my response while Poseidon wept on
him with such ferocious power.

I say to him, "come in, dear, I cannot hear you!"
\stop

\stage \eliza chuckles.

\startvernet
When he steps inside, he restates his question.

I say, "both of my rooms are occupied."

But your uncle was not dissuaded by this.

By the heaviness of his breath, the pain in his eye~- a sudden twist of his
mouth, as if trying to hide an offensive inquiry that was stuck between his
teeth.

I see in him a man in true need of shelter.

And I say this to him: "my dear, this weather is terrible, if you need my
shelter, you may have it."

\beat

He then apologies! I know not what for!
\stop

\stage \eliza chuckles.

\startvernet
He tells me that he has learned that his brother has been named an enemy of the
revolution - a threat to public safety.

And that he had seen proof that his brother's sentence was to be the
guillotine.

\beat

Then, with great sorrow, he asks me if I might shelter this outlaw.

\beat

He says to me that doing so shall make me an outlaw.

That if he is ever discovered, that I too shall meet God through a blade.
\stop

\stage Pause.

\startvernet
I say to him, "is he virtuous?"
\stop

\stage Pause.

\startvernet
"The most virtuous man I have ever known," is his reply.

\beat

But, dear, such times would make you believe Paris had descended into hell.

Under such circumstances, these words should certainly make me anxious.

But you see, it was not the words that touched me, it was the moment in his
eyes when he spoke it.

Your uncle certainly felt the cold flames of hell when he had first presented
himself to me, but in this glance I see in him... a moment of respite.

\beat

I say, "let him come."
\stop

\stage Pause.

\startvernet
That evening, your father steps through my door, unaware that he had just
stepped into a cell that he shall not step out of for the next nine months.

He comes in soaked by the storm and hardly lifts his head to greet me.

When I looked in his eyes, they were always somewhere else, rarely did he ever
make eye contact with me.

When I spoke to him, he would reply succinctly with a cold, unfeeling
intonation.

Hardly anything like the heroic outlaw I had hoped for.

\beat

I tell him that I house two other men with me, each belonging to the two
bedrooms available for rent.

But I may yet shelter him in the third bedroom that I shall fashion for him.

\beat

"As long as I have access to a light," is all he replies.
\stop

\starteliza
Who were the other two?
\stop

\startvernet
First was my cousin, \sarret.

He stayed with me while he was studying to become a geometer.

The other was a former professor of mathematics and a montagnard by creed.

His name was \marcoz.

\beat

The former knew your father by reputation.

The latter recognized him on sight.
\stop

\starteliza
A montagnard?
\stop

\startvernet
Oh, yes. But he was not a worry. I had a chat with him.

His connection to the party was what allowed your father to receive news of
the convention's doings while he remained hidden beneath its shadow.
\stop

\starteliza
He swore himself to espionage against his party? Why?
\stop

\stage \vernet smirks.

\startvernet
Because he chose me, dear.
\stop

\starteliza
Oh.
\stop

\startvernet
You should consider becoming a widow, darling. It's glamorous.
\stop

\stage \eliza chuckles.

\starteliza
No.
\stop

\stage \vernet laughs.

\startvernet
Your father didn't mind the arrangement.

And he said it was valuable to learn what decisions were being made at the
\convention.

\beat

I ask him: "why?"

And all he does is pout before retreating into his quarters.

\beat

"Because they're MONSTERS," I hear him cry!
\stop

\stage Pause.

\startvernet
\cabanis would come to visit on a weekly basis to deliver correspondence from
your mother, and each visit your father would have written to her six times
over.

This continued until one night I receive another knock on this door.

When before me stood a hooded rogue who, in a hush, introduced herself to be
\sophie[condorcet].

Of course, I welcome her.

\beat

When she revealed her face to me, she had a fierce look in her eyes, as if she
was silently, motionlessly rehearsing lines from a play.

"Something is wrong with my husband," she says.

\beat

"What is this?" I ask. "He has been in silence all these past two months."

And she shakes her head at this.

\beat

"It is some {\emph affliction} of the mind," she replies!

\beat

I ask: "what remedy do you intend for him?"

\beat

She looks into me and says, "I must listen to him. I must do so no matter how
much pain his words will cause me."

\beat

"And when the moment is right, I shall resolve this with reason."
\stop

\stage Pause.

\startvernet
Dear... I must know. Do you have any idea what she meant by that? Reason?
\stop

\starteliza
By reason? I don't follow.
\stop

\startvernet
Surely, you must know what she had said to him that night.
\stop

\starteliza
Well, she had shared with me some detail about this moment, but she refused to
tell me what exactly she had said to him.

She insisted that the words she chose that night were the culmination of six
weeks of careful planning.

That if she simply repeated them, then they would mean nothing to me.
\stop

\startvernet
And yet, you had no tenacity? How could you have sat idle and not pressed the
matter?
\stop

\starteliza
What excites you so much about this?
\stop

\startvernet
Because, my dear, when I next saw your father, I witnessed him transform into
the most hopeful human being I have ever known.

I must know what words could possibly have such power over the human spirit!

\beat

With a childlike wonder in his eyes, he shares with me that he had made some
great discovery:

that it was possible to construct an automaton capable of reason and that such
a machine could be used to verify the correctness of our laws based on the laws
we discover from nature.

\beat

I ask him, "how is such a thing possible?"

\beat

"Ah!" He says! "By first constructing the characteristica univeralis!"

\beat

I know not what he means!

\beat

He tells me that it is a universal calculus capable of expressing any thought
that is reachable through reason.

\beat

I know not what he means!

Then he suddenly begins cackling like a devil!

The first he had laughed since his arrival!

He says to me that he must first create such a thing before it can really begin
to make sense.

\beat

And of this, I finally know his meaning.

Next, I hear him call to \sarret, who he asks to bring a number of tomes from
the Academy~- a detailed list was procured from his pocket.

And after your father mentioned the name \leibniz to \sarret, \marcoz attempted
to warn him with the hypothesis that your father had succumb to mania.

I do not know why! And I silenced him!

\beat

For the next five weeks I watch your father work on his new calculus.

Not one hour was wasted~- though, alas, he had little else to occupy him!

\beat

His universal language soon covered the matters of physics, chemistry, and
finance~- and had done so with relative ease, as far as he tells me.

But now he says that he has reached another matter~- the matter of social
choices, he calls it.

He tells me that despite the definitions being by far the simplest, that
mathematics was not yet capable of reasoning about it, and that this was
unlikely to change within his remaining lifetime.
\stop

\stopcomponent
