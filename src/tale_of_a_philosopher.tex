\startcomponent tale_of_a_philosopher

\scene

\open \eliza is at a desk with the ten epochs of the tableau laid out before
her. By her is the same fireplace, lit as before. There is no longer any wind.

\stage \eliza begins reading the first epoch of the tableau.

\stage Enter \condorcet.

\startcondorcet
All human beings are born with the ability to receive sensations.

To perceive them and to distinguish them.

To remember, recognize~--
\stop

\stage Enter \leibniz and \voltaire.

\startsplit

\startleibniz
-- and combine them.
\stop

\split

\startvoltaire
-- and combine them.
\stop

\stopsplit

\startvoltaire
The mind is shaped by its senses.
\stop

\startcondorcet
It is shaped by its body.
\stop

\startleibniz
And it is motivated by its ability to reason about these things at once.
\stop

\startvoltaire
By reasoning about the shape of one's body, they can compel it to move.
\stop

\startleibniz
By reasoning about the shape of one's thoughts, they construct many monads that
help them construct monads!
\stop

\stage \eliza groans and stops reading.

\direction Lights down on \leibniz and \voltaire.

\stage \condorcet takes a seat by the fire. \eliza watches him.

\stage Pause. Silence.

\stage \eliza starts counting pages, gauging how long this is going to take...

\starteliza
First epoch: humans are united in tribes...
\stop

\stage Pause.

\starteliza
Second epoch: the transition to agriculture...
\stop

\stage Pause

\starteliza
The progress of agricultural techniques and the invention of alphabets...
\stop

\stage Pause.

\starteliza
The invention of geometry until the schism of the sciences around the time of
Alexander the Great...
\stop

\stage Pause.

\stage \eliza gestures in habit, but chooses silence instead.

\stage Pause.

\stage Pause.

\stage Pause. A sigh this time.

\starteliza
The invention of the printing press...
\stop

\stage Pause.

\stage Pause.

\stage \eliza separates out this last epoch. She begins to skim it...

\stage She then changes her mind, she instead wishes to see the last page of
the last epoch.

\starteliza
How consoling for the philosopher who laments the errors, the crimes, the
injustices~- which still pollute the Earth~- is this view of the human race:

emancipated from its shackles;

released from the empire of chance, built by the enemies of its progress;

advancing with a firm and sure step along the path of truth, virtue, and
happiness.

\beat

It is the contemplation of this world that rewards him for all his efforts to
assist the progress of reason and the defense of liberty.

He dares to link these strivings to the eternal chain of human destiny, and in
this persuasion, he learns the true reward of virtue:

the pleasure of having done some lasting good which fate can no longer destroy
by a sinister stroke of revenge.

\beat

Such contemplation is for him a refuge where the memory of his persecutors can
no longer pursue him.

There he lives in thought.

With men and women restored to their natural rights and dignity.

Where the suffering and corruption of greed, envy, or fear has long been
forgotten.

\beat

There he truly lives with his peers in the Elysium that his reason has created
for him, and that his love for humanity enhances with the purest joys.
\stop

\stage Pause. Silence.

\stage \eliza takes a deep breath.

\stopcomponent
