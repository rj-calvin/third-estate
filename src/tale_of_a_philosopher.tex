\startcomponent tale_of_a_philosopher

\scene

\open The \vernet residence living room.

\stage Enter \vernet.

\stage She sets the table with a breakfast of a twelve-egg omelette and
coffee.

\stage Enter \eliza.

\startvernet
Good evening, dear! I've prepared you breakfast!
\stop

\starteliza
It's near daybreak.
\stop

\startvernet
You're mistaken, it's near sunset.
\stop

\startvernet
Dear, you look like you are still living in yesterday.

I can see it in your sunken eyes.

\beat

Oh! What am I saying.

\beat{embraces \eliza}

I shouldn't make such comments about a woman's appearance.

I apologize, darling. I made you breakfast.
\stop

\stage \eliza seats herself.

\starteliza
This omelette is enormous. How many eggs did you put in it?
\stop

\startvernet
Twelve.
\stop

\starteliza
What? Twelve?

\beat{chuckles}

My attention must be worth a fortune.
\stop

\startvernet
It is to me, dear. Even if through such tired eyes.
\stop

\stage \eliza begins her meal by taking a sip of coffee.

\startvernet
I wish to apologize for my mood at where we had left off in my tale.

\beat

I have reflected on this matter and have decided that its cause was a selfish
desire to acquire knowledge that is forbidden to me.

Please understand, dear: I am an old woman.

Many of my days are occupied in my fantasy of completing the stories most dear
to my heart.

Thus, I reacted with hostility when your story failed to live up to these
expectations.

\beat

I apologize, dear.
\stop

\stage \eliza smiles.

\stage \vernet smiles, then takes a seat. She begins stuffing a pipe with
tobacco.

\starteliza
Please, madame. I don't mind displays of passion when they are honest.

In fact, I hope to experience whatever passions remain from your story~-
incomplete as it may be.
\stop

\startvernet
\beat{while lighting her pipe}

Oh, yes, dear. You have a debt to me.
\stop

\stage \vernet chuckles, puffing up the room with smoke signals.

\stopcomponent
