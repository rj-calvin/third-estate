\startcomponent et_toi_robespierre

\scene

\open Lights down on all but \condorcet.

\stage Enter \voltaire.

\startvoltaire
Professor. Do you mind if I consult you?
\stop

\stage Enter \leibniz.

\stage Pause.

\startvoltaire
I believe that I have discovered a proof, and I would like you to check my
work.
\stop

\startleibniz
Very well.
\stop

\goodbreak
\stage \voltaire smiles and wheels in a blackboard:

\startstaticMPfigure{circleA}
draw fullcircle scaled 18pt;
label(textext("A"), origin);
\stopstaticMPfigure

\startstaticMPfigure{circleC}
draw fullcircle scaled 18pt;
label(textext("C"), origin);
\stopstaticMPfigure

\startstaticMPfigure{circleAorC}
draw fullcircle yscaled 22pt xscaled 44pt;
label(textext("A | C"), origin);
\stopstaticMPfigure

\setupTABLE[
	frame=off,
	split=no,
	width=.111\textwidth,
	toffset=4pt,
	loffset=5mm,
]

\setupTABLE[row][first][
	bottomframe=on,
	toffset=0mm,
	boffset=4pt,
	loffset=4mm,
]

\startformula[modern]
\bTABLE

\bTR
\bTD[loffset=14pt] $1$ \eTD
\bTD[loffset=14pt] $2$ \eTD
\bTD[loffset=14pt] $3$ \eTD
\bTD[loffset=14pt] \bf\unknown \eTD
\bTD $721$ \eTD
\bTD[bottomframe=off] \rightarrow \eTD
\bTD[bottomframe=off,loffset=0pt]
\vbox{\vskip-2pt\usestaticMPfigure[circleA]}
\eTD
\eTR

\bTR
\bTD A \eTD
\bTD A \eTD
\bTD A \eTD
\bTD \bf\unknown \eTD
\bTD A \eTD
\eTR

\bTR
\bTD B \eTD
\bTD B \eTD
\bTD B \eTD
\bTD \bf\unknown \eTD
\bTD B \eTD
\eTR

\bTR
\bTD C \eTD
\bTD C \eTD
\bTD C \eTD
\bTD \bf\unknown \eTD
\bTD C \eTD
\eTR

\eTABLE
\stopformula

\startvoltaire
Suppose we have - oh, I don't know, seven-hundred twenty-one representatives
who each rank their preferences for three choices: A, B, and C.

However, unlike before, I have no intention of ascribing any interpretation to
these alternatives - they are totally arbitrary.

Nonetheless, we can assume that any decision function would select choice A if
every participant ranks A highest in their preferences, correct?
\stop

\startleibniz
Ah, yes, support for unanimity is essential for all decision functions.
\stop

\startvoltaire
And if our decision function is strategy-proof, then there should be no
incentive for any representative to withhold their true preferences.
\stop

\startleibniz
That is correct.
\stop

\startvoltaire
So then, say, if representative one were to change their preference for C, it
should therefore not be to their detriment to increase their ranking of C
over B.
\stop

\goodbreak
\stage \voltaire edits:

\startformula[modern]
\bTABLE

\bTR
\bTD[loffset=14pt] $1$ \eTD
\bTD[loffset=14pt] $2$ \eTD
\bTD[loffset=14pt] $3$ \eTD
\bTD[loffset=14pt] \bf\unknown \eTD
\bTD $721$ \eTD
\bTD[bottomframe=off] \rightarrow \eTD
\bTD[bottomframe=off,loffset=0pt]
\vbox{\vskip-2pt\usestaticMPfigure[circleA]}
\eTD
\eTR

\bTR
\bTD A \eTD
\bTD A \eTD
\bTD A \eTD
\bTD \bf\unknown \eTD
\bTD A \eTD
\eTR

\bTR
\bTD C \eTD
\bTD B \eTD
\bTD B \eTD
\bTD \bf\unknown \eTD
\bTD B \eTD
\eTR

\bTR
\bTD B \eTD
\bTD C \eTD
\bTD C \eTD
\bTD \bf\unknown \eTD
\bTD C \eTD
\eTR

\eTABLE
\stopformula

\startvoltaire
In other words, the outcome should remain A since if this edit changed the
outcome, then representative one would not be incentivized to report their true
preference since A is of a higher rank than either B or C.
\stop

\startleibniz
Yes, correct.
\stop

\startbarere
Gentlemen... We shall now begin the nominal call on the sentence of \louis.
\stop

\startlehardi
I motion that the sentence of death should not be passed with any less than a
two-thirds majority!
\stop

\startlajuinais
I declare that it should be made three-fourths!
\stop

\stage An uproar.

\startbarere
GENTLEMEN! ORDER! The nominal call shall proceed without modification.
\stop

\startvoltaire
However, the case would certainly be different should representative one modify
their preference for A relative to either B or C, correct?
\stop

\startleibniz
Yes. Since we are assuming an arbitrary decision function, we can't assume that
lowering the rank of A for any representative will preserve the outcome we
obtained by unanimity.
\stop

\startvoltaire
But we do know that representative one cannot be punished for reporting an
increased preference for C.

We don't know whether it will be certain to modify the outcome, but we do know
that if it did, the outcome could only be C.
\stop

\stage \voltaire edits:

\startformula[modern]
\bTABLE

\bTR
\bTD[loffset=14pt] $1$ \eTD
\bTD[loffset=14pt] $2$ \eTD
\bTD[loffset=14pt] $3$ \eTD
\bTD[loffset=14pt] \bf\unknown \eTD
\bTD $721$ \eTD
\bTD[bottomframe=off] \rightarrow \eTD
\bTD[bottomframe=off,loffset=0pt]
\vbox{\vskip-2pt\usestaticMPfigure[circleAorC]}
\eTD
\eTR

\bTR
\bTD C \eTD
\bTD A \eTD
\bTD A \eTD
\bTD \bf\unknown \eTD
\bTD A \eTD
\eTR

\bTR
\bTD A \eTD
\bTD B \eTD
\bTD B \eTD
\bTD \bf\unknown \eTD
\bTD B \eTD
\eTR

\bTR
\bTD B \eTD
\bTD C \eTD
\bTD C \eTD
\bTD \bf\unknown \eTD
\bTD C \eTD
\eTR

\eTABLE
\stopformula

\direction Each deputy is called to the tribune.

\startbarere
\mailhe!
\stop

\startmailhe
I do vote for death. However, I motion that should there be a majority, then
the \convention should decide whether or not his execution should be delayed
until the acceptance of the new constitution.
\stop

\stage Loud groans.

\startleibniz
The outcome is now ambiguous, but it is true that we know that no
strategy-proof decision function would select B in this case.

Otherwise, representative one would not be incentivized to report their
preference for C.
\stop

\startvoltaire
An outcome of B would be like Marie Antoinette swapping her preference to
chocolate cake when presented an irrelevant alternative.
\stop

\startleibniz
As an analogy, but not as a formal argument.
\stop

\startvoltaire
Ah, there is time yet for formal arguments.

Indeed, I now wish to argue this: that we may now repeat this procedure for
representatives two, three, and so on.
\stop

\startbarere
\thuriot!
\stop

\startthuriot
I vote death. {\emph Immediate} death. Let the tyrant carry his head to the
scaffold.
\stop

\startleibniz
How would you perform this algorithm?
\stop

\startvoltaire
Simple. We proceed conditionally on whether the outcome is changed by the
increase in C over A.

If the outcome remains the same, then we continue to perform the same
operations, following the same reasoning already demonstrated, on
representative two.

Then, if the procedure again fails to change the outcome, we continue to
representative three, and then so on.

If, at any point, the outcome is changed to C, we shall halt on the
representative responsible for it.
\stop

\startleibniz
Ah! I see your reasoning. This algorithm is sure to terminate by unanimity.

For in the worst case, you would modify the preference of representative
seven-hundred twenty-one, and by then the choice of C will be placed at the top
of every ranking.

By unanimity, the outcome must be C in such a case.
\stop

\startbarere
\vergniaud!
\stop

\startvergniaud
I have fought for an appeal to the people.

Yet, the \convention has decided otherwise.

I have failed to uphold my principles, and so I have nothing more to say.

\beat

I decided that \louis was guilty, thus I open the penal code to see that the
penalty is death.

I, therefore, pronounce death upon him, but do so with the restriction proposed
by \mailhe.
\stop

\startvoltaire
Given now that we know the existence of a representative where, upon increasing
their rank for C results in the outcome changing to C.

We shall call this representative the pivotal voter.

I will label the index of the pivotal voter's location with the letter N.
\stop

\stage \voltaire flips the board, revealing two tables:

\startformula[modern]
\bTABLE

\bTR
\bTD[loffset=14pt] $1$ \eTD
\bTD[loffset=14pt] \bf\unknown \eTD
\bTD[loffset=9pt] \bf\unknown \eTD
\bTD[loffset=14pt] $N$ \eTD
\bTD[loffset=9pt] \bf\unknown \eTD
\bTD[loffset=14pt] \bf\unknown \eTD
\bTD $721$ \eTD
\bTD[bottomframe=off] \rightarrow \eTD
\bTD[bottomframe=off,loffset=0pt]
\vbox{\vskip-2pt\usestaticMPfigure[circleA]}
\eTD
\eTR

\bTR
\bTD C \eTD
\bTD \bf\unknown \eTD
\bTD C \eTD
\bTD A \eTD
\bTD A \eTD
\bTD \bf\unknown \eTD
\bTD A \eTD
\eTR

\bTR
\bTD A \eTD
\bTD \bf\unknown \eTD
\bTD A \eTD
\bTD C \eTD
\bTD B \eTD
\bTD \bf\unknown \eTD
\bTD B \eTD
\eTR

\bTR
\bTD B \eTD
\bTD \bf\unknown \eTD
\bTD B \eTD
\bTD B \eTD
\bTD C \eTD
\bTD \bf\unknown \eTD
\bTD C \eTD
\eTR

\eTABLE
\stopformula

\blank[2*line]

\startformula[modern]
\bTABLE

\bTR
\bTD[loffset=14pt] $1$ \eTD
\bTD[loffset=14pt] \bf\unknown \eTD
\bTD[loffset=9pt] \bf\unknown \eTD
\bTD[loffset=14pt] $N$ \eTD
\bTD[loffset=9pt] \bf\unknown \eTD
\bTD[loffset=14pt] \bf\unknown \eTD
\bTD $721$ \eTD
\bTD[bottomframe=off] \rightarrow \eTD
\bTD[bottomframe=off,loffset=0pt]
\vbox{\vskip-2pt\usestaticMPfigure[circleC]}
\eTD
\eTR

\bTR
\bTD C \eTD
\bTD \bf\unknown \eTD
\bTD C \eTD
\bTD C \eTD
\bTD A \eTD
\bTD \bf\unknown \eTD
\bTD A \eTD
\eTR

\bTR
\bTD A \eTD
\bTD \bf\unknown \eTD
\bTD A \eTD
\bTD A \eTD
\bTD B \eTD
\bTD \bf\unknown \eTD
\bTD B \eTD
\eTR

\bTR
\bTD B \eTD
\bTD \bf\unknown \eTD
\bTD B \eTD
\bTD B \eTD
\bTD C \eTD
\bTD \bf\unknown \eTD
\bTD C \eTD
\eTR

\eTABLE
\stopformula

\startbarere
\robespierre!
\stop

\startrobespierre
I vote death for the tyrant.
\stop

\startvoltaire
Here we see the moment before and after the pivotal voter lifts their
preference for C above A.

\beat

Now, I'd like to apply a new algorithm to this second table - the one taking
place after C had become the new result.

Notice that the representatives to the right of the pivotal voter currently
list C at the bottom of their preferences.

Therefore, changing their preference for B and lifting it above A should
not change the outcome.
\stop

\startleibniz
True. If any of these actions taken by the right voters changed the outcome,
they would be compelled to to do it no matter their true preferences for B and
A.
\stop

\goodbreak
\stage \voltaire edits:

\startformula[modern]
\bTABLE

\bTR
\bTD[loffset=14pt] $1$ \eTD
\bTD[loffset=14pt] \bf\unknown \eTD
\bTD[loffset=9pt] \bf\unknown \eTD
\bTD[loffset=14pt] $N$ \eTD
\bTD[loffset=9pt] \bf\unknown \eTD
\bTD[loffset=14pt] \bf\unknown \eTD
\bTD $721$ \eTD
\bTD[bottomframe=off] \rightarrow \eTD
\bTD[bottomframe=off,loffset=0pt]
\vbox{\vskip-2pt\usestaticMPfigure[circleC]}
\eTD
\eTR

\bTR
\bTD C \eTD
\bTD \bf\unknown \eTD
\bTD C \eTD
\bTD C \eTD
\bTD B \eTD
\bTD \bf\unknown \eTD
\bTD B \eTD
\eTR

\bTR
\bTD A \eTD
\bTD \bf\unknown \eTD
\bTD A \eTD
\bTD A \eTD
\bTD A \eTD
\bTD \bf\unknown \eTD
\bTD A \eTD
\eTR

\bTR
\bTD B \eTD
\bTD \bf\unknown \eTD
\bTD B \eTD
\bTD B \eTD
\bTD C \eTD
\bTD \bf\unknown \eTD
\bTD C \eTD
\eTR

\eTABLE
\stopformula

\startbarere
\marat!
\stop

\startmarat
Death.
\stop

\startvoltaire
Likewise, we can do a similar operation to the representatives to the left of
the pivotal voter.

Namely, we can increase their preferences for B and lift it above A since C is
already at the top of their preferences.
\stop

\goodbreak
\stage \voltaire edits:

\startformula[modern]
\bTABLE

\bTR
\bTD[loffset=14pt] $1$ \eTD
\bTD[loffset=14pt] \bf\unknown \eTD
\bTD[loffset=9pt] \bf\unknown \eTD
\bTD[loffset=14pt] $N$ \eTD
\bTD[loffset=9pt] \bf\unknown \eTD
\bTD[loffset=14pt] \bf\unknown \eTD
\bTD $721$ \eTD
\bTD[bottomframe=off] \rightarrow \eTD
\bTD[bottomframe=off,loffset=0pt]
\vbox{\vskip-2pt\usestaticMPfigure[circleC]}
\eTD
\eTR

\bTR
\bTD C \eTD
\bTD \bf\unknown \eTD
\bTD C \eTD
\bTD C \eTD
\bTD B \eTD
\bTD \bf\unknown \eTD
\bTD B \eTD
\eTR

\bTR
\bTD B \eTD
\bTD \bf\unknown \eTD
\bTD B \eTD
\bTD A \eTD
\bTD A \eTD
\bTD \bf\unknown \eTD
\bTD A \eTD
\eTR

\bTR
\bTD A \eTD
\bTD \bf\unknown \eTD
\bTD A \eTD
\bTD B \eTD
\bTD C \eTD
\bTD \bf\unknown \eTD
\bTD C \eTD
\eTR

\eTABLE
\stopformula

\startbarere
\paine!
\stop

\startpaine
I --
\stop

\startmarat
-- He is a Quaker! He should be forbidden from voting since he is biased by his
religion!
\stop

\stage Groans from the \convention.

\startpaine
Exile.
\stop

\startvoltaire
So, here we reach a key moment of my argument - it is essential in order to
arrive at my eventual destination.

I would like to argue that if the pivotal voter now swaps their preferences of
A and C, then the outcome will change to A.
\stop

\startleibniz
Hold on, \voltaire! You can't make such a leap in logic. You have modified the
rank of A in many positions, so there is no guarantee that this pivotal voter
has the same influence they once did following your first algorithm.
\stop

\startvoltaire
Indeed. But you must confess that we still know that such a change will either
do nothing, or it will change the outcome to A.
\stop

\startleibniz
Now, it would seem we are stuck as there is no longer any means to decide
between the two possibilities.
\stop

\startvoltaire
Not so.
\stop

\goodbreak
\stage \voltaire edits:

\startformula[modern]
\bTABLE

\bTR
\bTD[loffset=14pt] $1$ \eTD
\bTD[loffset=14pt] \bf\unknown \eTD
\bTD[loffset=9pt] \bf\unknown \eTD
\bTD[loffset=14pt] $N$ \eTD
\bTD[loffset=9pt] \bf\unknown \eTD
\bTD[loffset=14pt] \bf\unknown \eTD
\bTD $721$ \eTD
\bTD[bottomframe=off] \rightarrow \eTD
\bTD[bottomframe=off,loffset=0pt]
\vbox{\vskip-2pt\usestaticMPfigure[circleAorC]}
\eTD
\eTR

\bTR
\bTD C \eTD
\bTD \bf\unknown \eTD
\bTD C \eTD
\bTD A \eTD
\bTD B \eTD
\bTD \bf\unknown \eTD
\bTD B \eTD
\eTR

\bTR
\bTD B \eTD
\bTD \bf\unknown \eTD
\bTD B \eTD
\bTD C \eTD
\bTD A \eTD
\bTD \bf\unknown \eTD
\bTD A \eTD
\eTR

\bTR
\bTD A \eTD
\bTD \bf\unknown \eTD
\bTD A \eTD
\bTD B \eTD
\bTD C \eTD
\bTD \bf\unknown \eTD
\bTD C \eTD
\eTR

\eTABLE
\stopformula

\startbarere
\danton!
\stop

\startdanton
Death...
\stop

\startvoltaire
Let's consider the case where the outcome remains C.

Then observe that, if this were the case, we would then be allowed to lift
every A to the left and right of the pivotal voter.

If we did this, we would be able to construct the exact table as the one
immediately prior to the change induced by the pivotal voter, where - you would
recall - we observed that the outcome was A.
\stop

\startleibniz
AH! I see!

Thus, the case of C allows you to construct a contradiction.

This case is therefore impossible: the outcome must be A!

Very clever!
\stop

\startvoltaire
Me, clever? Why thank you, professor.
\stop

\goodbreak
\stage \voltaire edits:

\startformula[modern]
\bTABLE

\bTR
\bTD[loffset=14pt] $1$ \eTD
\bTD[loffset=14pt] \bf\unknown \eTD
\bTD[loffset=9pt] \bf\unknown \eTD
\bTD[loffset=14pt] $N$ \eTD
\bTD[loffset=9pt] \bf\unknown \eTD
\bTD[loffset=14pt] \bf\unknown \eTD
\bTD $721$ \eTD
\bTD[bottomframe=off] \rightarrow \eTD
\bTD[bottomframe=off,loffset=0pt]
\vbox{\vskip-2pt\usestaticMPfigure[circleA]}
\eTD
\eTR

\bTR
\bTD C \eTD
\bTD \bf\unknown \eTD
\bTD C \eTD
\bTD A \eTD
\bTD B \eTD
\bTD \bf\unknown \eTD
\bTD B \eTD
\eTR

\bTR
\bTD B \eTD
\bTD \bf\unknown \eTD
\bTD B \eTD
\bTD C \eTD
\bTD A \eTD
\bTD \bf\unknown \eTD
\bTD A \eTD
\eTR

\bTR
\bTD A \eTD
\bTD \bf\unknown \eTD
\bTD A \eTD
\bTD B \eTD
\bTD C \eTD
\bTD \bf\unknown \eTD
\bTD C \eTD
\eTR

\eTABLE
\stopformula

\startbarere
\legendre!
\stop

\startlegendre
DEATH TO THE OPPRESSOR! Let his corpse be divided into eighty-four pieces and
distributed to each department so that they may burn the meat under the trees
of liberty!
\stop

\startvoltaire
Now, we can lift all of the Bs to the left of and including the pivotal voter.
\stop

\goodbreak
\stage \voltaire edits:

\startformula[modern]
\bTABLE

\bTR
\bTD[loffset=14pt] $1$ \eTD
\bTD[loffset=14pt] \bf\unknown \eTD
\bTD[loffset=9pt] \bf\unknown \eTD
\bTD[loffset=14pt] $N$ \eTD
\bTD[loffset=9pt] \bf\unknown \eTD
\bTD[loffset=14pt] \bf\unknown \eTD
\bTD $721$ \eTD
\bTD[bottomframe=off] \rightarrow \eTD
\bTD[bottomframe=off,loffset=0pt]
\vbox{\vskip-2pt\usestaticMPfigure[circleA]}
\eTD
\eTR

\bTR
\bTD B \eTD
\bTD \bf\unknown \eTD
\bTD B \eTD
\bTD A \eTD
\bTD B \eTD
\bTD \bf\unknown \eTD
\bTD B \eTD
\eTR

\bTR
\bTD C \eTD
\bTD \bf\unknown \eTD
\bTD C \eTD
\bTD B \eTD
\bTD A \eTD
\bTD \bf\unknown \eTD
\bTD A \eTD
\eTR

\bTR
\bTD A \eTD
\bTD \bf\unknown \eTD
\bTD A \eTD
\bTD C \eTD
\bTD C \eTD
\bTD \bf\unknown \eTD
\bTD C \eTD
\eTR

\eTABLE
\stopformula

\startleibniz
One moment, please.
\stop

\startvoltaire
Hm? What is it?
\stop

\startleibniz
Oh. I just need a moment to ponder.
\stop

\stage Pause.

\startbarere
\serres!
\stop

\startserres
\beat{sobbing}

What cruelty have I seen! How evil is this fate for France! How ashamed I am to
be witness to the defilement of my country!
\stop

\startdanton
Get on with it! No more speeches, damn it!
\stop

\startserres
I... I... I vote that the king may live! That he may be spared such barbarity!

But that he still be properly punished for his misdeeds and misgivings.

I therefore vote for his exile, but with the condition that he remain
imprisoned until the conclusion of our war with Austria.
\stop

\stage Groans from the \convention.

\startleibniz
I'm prepared to continue.
\stop

\startvoltaire
Excellent!

Suppose now that I produce another modification to the rankings of A.

This time I shall do so for the representatives to the right of the pivotal
voter.

Again, this shall require us to decide whether or not this change will also
change the outcome.
\stop

\goodbreak
\stage \voltaire edits:

\startformula[modern]
\bTABLE

\bTR
\bTD[loffset=14pt] $1$ \eTD
\bTD[loffset=14pt] \bf\unknown \eTD
\bTD[loffset=9pt] \bf\unknown \eTD
\bTD[loffset=14pt] $N$ \eTD
\bTD[loffset=9pt] \bf\unknown \eTD
\bTD[loffset=14pt] \bf\unknown \eTD
\bTD $721$ \eTD
\bTD[bottomframe=off] \rightarrow \eTD
\bTD[bottomframe=off,loffset=0pt]
\vbox{\vskip-2pt\usestaticMPfigure[circleAorC]}
\eTD
\eTR

\bTR
\bTD B \eTD
\bTD \bf\unknown \eTD
\bTD B \eTD
\bTD A \eTD
\bTD B \eTD
\bTD \bf\unknown \eTD
\bTD B \eTD
\eTR

\bTR
\bTD C \eTD
\bTD \bf\unknown \eTD
\bTD C \eTD
\bTD B \eTD
\bTD C \eTD
\bTD \bf\unknown \eTD
\bTD C \eTD
\eTR

\bTR
\bTD A \eTD
\bTD \bf\unknown \eTD
\bTD A \eTD
\bTD C \eTD
\bTD A \eTD
\bTD \bf\unknown \eTD
\bTD A \eTD
\eTR

\eTABLE
\stopformula

\startleibniz
Surely, at this stage the outcome must be B!
\stop

\startvoltaire
B? You forget yourself, professor! Such an outcome would not allow the decision
function to be strategy-proof.
\stop

\startleibniz
Right, it must, therefore, be C! Or perhaps it is indeterminate!
\stop

\startvoltaire
But is it so?
\stop

\stage \leibniz contemplates.

\startleibniz
Ah! It must be C!

Notice that every representative apart from the pivotal voter has A listed at
the bottom of their preferences.

Therefore, if the outcome were to remain A in this case, then this would imply
that the outcome is determined exclusively by the preference of the pivotal
voter.

This would contradict our assumption that our decision function is non-trivial!
\stop

\startvoltaire
I made no such assumption.
\stop

\startbarere
\brissot!
\stop

\startbrissot
I am convinced that some evil genius is responsible for the failure of the
appeal to the people.

And I am convinced that the consequences of this failure shall forever condemn
this \convention to be remembered as monsters.

I do believe in death for \louis, but not at the cost of the people's
sovereignty and the dignity of their leaders.

As such, I uphold the condition of \mailhe.

I believe that respite until the completion of our constitution is perhaps the
final hope for France's redemption.
\stop

\startvoltaire
So, A or C? Again, we are faced with the dilemma.

\beat

Observe that B is ranked above C by every representative.

Thus, if the outcome were C, it must remain C even if B were to be lifted by
the pivotal voter.

But then this would contradict our assumption that all decision functions
support unanimity.

Therefore, the outcome must be --
\stop

\startleibniz
-- NO-O-O! NO-O!

\beat

ALL STRATEGY-PROOF DECISION FUNCTIONS ARE TRIVIAL! NO-O-O!
\stop

\startbarere
\condorcet!
\stop

\startvoltaire
DO WE LIVE IN THE BEST OF ALL POSSIBLE WORLDS, PROFESSOR?
\stop

\startleibniz
WE MUST! IT MUST BE SO!
\stop

\startvoltaire
DO YOU INSIST ON IT!? PROVE IT! PROVE IT TO ME!
\stop

\startbarere
\condorcet!
\stop

\stage \condorcet finds himself on the tribune.

\stage \condorcet is frozen. The \convention stares at him.

\stage Pause.

\stage Pause. Whispers.

\stage \condorcet has trouble breathing.

\startsplit

\startanonymous
Et toi!?
\stop

\split
\leavevmode

\startbarere
Et toi!?
\stop

\stopsplit

\startsplit

\startthemountain
ET TOI!? ET TOI!?
\stop

\split

\startthemoderates
ET TOI!? ET TOI!?
\stop

\stopsplit

\starttheconvention
ET TOI!? ET TOI!? ET TOI!?
\stop

\startcondorcet
I...
\stop

\direction \thestranger whispers to \condorcet.

\startthestranger
Et toi, \robespierre?
\stop

\startcondorcet
The constitution requires that \louis be tried as a citizen.

I... have acted as judge, thus I am not a legislator.

The penal code requires the death of traitors, and \louis has proven to be one.

\beat{resolute}

However! The death penalty is against my principles, no matter the charge
against the defendant.

I, therefore, must oppose the death penalty; and, t-therefore, shall vote for
the second most severe penalty, next to death.

\beat

W-which is that \louis shall be sentenced to a life of forced labor.
\stop

\startbrissot
WHAT!? You would have him {\emph enslaved}!? {\WORDS\condorcet}, HOW COULD
YOU!?
\stop

\startcondorcet
Wha(t) - I...
\stop

\stage The \mountain chatters in confusion while the \moderates make discordant
accusations toward \condorcet.

\startbarere
Silence! Quiet! The nominal call is done with!
\stop

\stage The secretary~- a man by the name of \camus~- tallies the vote.

\stage \camus concludes his arithmetic, then shakes his head in disbelief.

\startcamus
Pardon me, messieurs, I must recount.
\stop

\stage Nobody reacts.

\stage \camus concludes his arithmetic.

\startcamus
In the name of the \convention[national], by a margin of one vote, the sentence
to be pronounced on the accused, \louis[capet], is that of death.
\stop

\stage Emotions run wild! Tears are shed, sobs are heard... And a cry of united
victory.

\direction \thestranger whispers to \condorcet.

\stage Many of the \moderates simultaneously call for the tribune.

\stage \robespierre, seeing this, rushes to the floor.

\startrobespierre
CITIZE-E-ENS! I know what animates you! It is clear what compels you to the
tribune!

But know that the authority of this \convention depends on this decision being
final!

The safety of the nation is contingent on~--!
\stop

\startcondorcet
-- I DEMAND THE FLOOR, {\WORD\barere[citizen]}!
\stop

\stage Pause. Silence.

\stage \condorcet[therevolutionary] ascends the tribune.

\stage Pause. Silence.

\startcondorcet
T-today...

\beat{clears throat}

T-today, kings across the world strive to have us depicted as... as savages for
having defied our former tyrant.

But now that we've decided it's finally time to continue voting, why don't we
take this opportunity to - to demonstrate for these spectators our humanity?

\beat

Let us, therefore, vote without delay! Let us begin by abolishing the death
penalty! Except in the case of kings!
\stop

\stage Pause. Silence.

\startcondorcet
A-after that, we could vote on the establishment of public education!

It. Is. IMPERATIVE that we ensure that all of France can be taught to reason,
so that every citizen in the nation may learn to resolve their conflicts
through the productive exchange and composition of their ideas!

And, of equal importance, that the population is also taught to engage this
same talent in writing to guarantee that {\emph every} human being is gifted
their ability to exercise their right to contribute to human history!

\beat

Citizens! Patriots! Friends! We are now free to eliminate all the laws that
exist to oppress debtors for the profit of creditors!

We can create institutions to facilitate the adoption of children who have been
orphaned by the war; improve the treatment of citizens born out of wedlock;
provide widows the right to vote; construct housing and distribute welfare to
wounded veterans!

Then... Then! If any despot still dares to reproach our republic for the
judgement of \louis, we may tell them: "we have killed one man so that we may
save one hundred thousand others!"
\stop

\stage Pause. Silence.

\stage \condorcet knows this silence well. He's known it his whole life.

\stage \barere begins to applaud - his mood having been marginally improved.

\stopcomponent
