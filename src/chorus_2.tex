\startcomponent chorus_2

\scene

\open \condorcet is revealed reading the Paris broadsheets. He imagines the
aftermath of the massacre beside him.

\stage Enter \leibniz.

\startleibniz
The horror! Oh, the horror! How could this happen?

\(deep breath)
We must remain calm! We must remain calm, and we must consider the causes.
\stop

\stage Enter \voltaire.

\startvoltaire
Ah, the answer is simple! The cause was the ignition of gunpowder within the
chambers of muskets!
\stop

\stage \voltaire surveys the corpses.

\startvoltaire[continued]
Hm. Well, I don't see any children, so I suppose it's not the worst of all
possible massacres!

\(beat)
Shame about the women, though.
\stop

\startleibniz
How could this have happened?
\stop

\startvoltaire
Oh, I'm sorry, was my assessment insufficient?
\stop

\startleibniz
Yes. You conflate a necessary truth with a contingent truth.
\stop

\startvoltaire
Elaborate.
\stop

\startleibniz
A necessary truth is that which must be true in all possible worlds.

As you have observed, in all worlds where gunpowder exists, there must be an
explosion when such gunpowder is ignited by a spark.

\(beat)
However, a contingent truth would be the gunmen pulling their triggers.

It is a truth, for it has been observed, but it is not so that in all worlds
where guns are wielded by men that their triggers must be pulled.

Since the acting of firing only occurs in few possible worlds, and not all of
them, we consider this truth to be contingent on some other causes.
\stop

\startvoltaire
Ah, so what you're saying is that this is all \lafayette's fault?

I knew it!
\stop

\startleibniz
Unfortunately, \voltaire, it is not so simple with contingent truths.

Notice that all necessary truths have a certain atomicity, a continuity, in the
way they manifest in the physical world.

This, I fear, is not so for contingent truths.

Instead, contingent truths are made true not by a single prior, but a series or
priors - a chain, or sequence, of events that are not necessarily linked in a
continuous fashion.

\(beat)
Notice that if \lafayette were responsible for the shooting, it would also be
necessary to inquire why \lafayette had arrived to the scene at all, then it
would be necessary to inquire why the initial lynching had occurred, or why the
petition was proposed - so on, so on.
\stop

\startvoltaire
Then what is the point of this exercise?
\stop

\startleibniz
The point is that, while lamentable that this series of contingent events
occurred, there is nonetheless a consistent {\underbar structure} to their
form!

\(beat)
Consider that we can quantify the relationship between prior events to the
event of ignition using probabilities.

We ask ourselves, by looking at the sequence of events, what the probability of
each event contributes to predicting the event of ignition.

By listing the probabilities, we can deduce the collection of primary causes,
then within those primary causes, we can repeat the same analysis to arrive at
an ever increasing perfection of understanding.
\stop

\startvoltaire
Well, what's keeping you then? Get to work! Leave me in peace!
\stop

\startleibniz
I only can describe the structure, I don't have the capacity to emulate the
calculations of God.
\stop

\startvoltaire
Then I repeat: what is the point!
\stop

\startleibniz
You are frustrated that we cannot achieve a perfect understanding.

I sympathize. However, the absence of perfection should not preclude our
striving for progress!

It's one step forward to a destination infinitely far away!
\stop

\stage \voltaire surveys the bodies once more.

\startvoltaire
Oh, I seem to have missed one.
\stop

\stage \voltaire pushes one of the bodies.

\startvoltaire[continued]
Ah! It appears there is a child in this mess. Not shot, but smothered by the
bosom of its mother's corpse!
\stop

\stopcomponent
