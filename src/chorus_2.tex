\startcomponent chorus_2

\scene

\direction \condorcet is revealed reading the daily broadsheets.

\open The altar of the Fatherland at the Champ de Mars. A crowd of \patriots,
including \marat, have gathered to sign a petition.

\stage A woman screams on top of the altar: there's someone below her!

\stage Two \patriots pull two \royalists out from beneath the altar.

\startpatriot[number=1]
Two royalists are trying to have a peep show!
\stop

\startpatriot[number=2]
They was hiding below the altar! With a little picnic and a bunch of tools!
\stop

\startmarat
WHAT KIND OF TOOLS?
\stop

\startpatriots
Tools? What tools?
\stop

\startpatriot[number=3]
They don't wear the cockade! They're royalists!
\stop

\startmarat
THEY'RE AUSTRIAN TERRORISTS!
\stop

\stage The \patriots gasp.

\startpatriot[number=3]
A bomb? There's a BOMB?
\stop

\startmarat
TWO HEADS are necessary for civility and virtue!
\stop

\stage The \patriots pin down the two \royalists and decapitate their heads.

\stage Enter a \militia, led by drums and \lafayette.

\stage The \militia unfurls a red flag.

\startmarat[continued]
\lafayette! Join us! We are excercising our right to assembly!
\stop

\startlafayette
Patriots! We have been sent by the municipality of Paris, on decree of the
National Assembly to ensure that the peace be maintained!

We understand --
\stop

\startpatriot[number=2]
-- shut up, \lafayette!
\stop

\startlafayette
Citizens! Please! You know me!
\stop

\stage The \patriots begin throwing rocks at \lafayette.

\startlafayette[continued]
Aim above their heads!
\stop

\stage The \militia takes aim.

\startlafayette[continued]
Fire!
\stop

\stage The \militia fires.

\startsplit

\startpatriot[number=2]
HE'S A TOOTHLESS DOG!
\stop

\split

\startpatriot[number=3]
THEY'RE FIRING BLANKS!
\stop

\stopsplit

\startlafayette
Citizens! Citizens! I beg for you to disperse!
\stop

\stage The \militia takes aim.

\stage One of the riflers is hit by a rock, causing his musket to fire on his
neighbor.

\stage The \militia fires.

\startlafayette[continued]
NO! Stop!
\stop

\stage The \patriots scream and disperse, leaving corpses behind.

\stage The \militia pursues. \lafayette retreats.

\stage \condorcet looks up from the broadsheet.

\direction \thestranger whispers to \condorcet.

\startcondorcet
H-how could \lafayette do this?
\stop

\stage Enter \leibniz.

\startleibniz
We must consider the causes!
\stop

\stage Enter \voltaire.

\startvoltaire
Ah, the answer is simple! The cause was the ignition of gunpowder!
\stop

\startcondorcet
He was my friend! Just fifteen days ago did he not laugh at my joke about
hereditary kings?
\stop

\startleibniz
We must identify the monads. There were many possible worlds along this
worldline!
\stop

\startvoltaire
Shut up, \leibniz.
\stop

\startleibniz
But if its a worldline, we must study it as a worldline.
\stop

\stage \voltaire surveys the corpses.

\startvoltaire
Tell me. What am I supposed to be looking for, doctor?
\stop

\startleibniz
A worldline is the path that is traveled by a point of reference in space and
time.

Worldlines are what we draw in the geometry of information!
\stop

\startvoltaire
Hm, well, I don't see any children.

So I suppose it's not the worst of all possible worldlines.

\(beat)
Real shame about the women, though.
\stop

\direction \thestranger whispers to \condorcet.

\startcondorcet
But, \leibniz, i-if worldlines can contain other worlds that contain
worldlines, is there a worldline that contains all worlds that contain all
worldlines?
\stop

\startleibniz
Ah, that would be the problem of worldcircles!
\stop

\startcondorcet
What's a worldcircle?
\stop

\startleibniz
A worldcircle is a category of monad!
\stop

\stopcomponent
