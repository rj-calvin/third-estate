\startcomponent three_body_problem

\scene

\open \eliza is alone at a desk. A small fire keeps the room warm despite the
wind howling beyond the walls.

\stage \eliza is organizing the documents she has collected, arranging them in
the order she plans to read them.

\stage \eliza chooses a document and begins.

\direction Lights up on \condorcet, who sits in a rocking chair. He writes by
the light of the flame.

\stage Enter \sophie.

\direction TODO

\startsophie
Such tragedy that the mind cannot observe itself!

For yours cannot see how it has changed!
\stop

\startcondorcet
What? No, that's not~--
\stop

\startsophie
-- Such tragedy! That the mind cannot observe others!

For yours cannot see that you are no longer the man I love!
\stop

\stage \condorcet is shocked!

\startcondorcet
WHAT!? NO! NO, NO! That can't be~--!
\stop

\startsophie
-- SUCH TRAGEDY! That the mind is not the true master of its fate!

For yours cannot control the laws that govern nature!
\stop

\stage \condorcet backs down.

\startsophie
My dear, C... You are too awfully aware of a recurring truth: that all
enlightenments are merely partial.

And I fear that I must tell you that your own enlightenment has reached its
limit.

Thus, I feel it as my duty to inform you that your mind is due... for another.
\stop

% \startsophie
% Woe unto the philosopher whose mind has been lost to the will of others.
%
% His reason compromised by the fickle passions of those minds that lack the
% strength and scope to derive the abstract and general truths of moral ideas.
%
% \beat
%
% So convinced is the philosopher that the reason he wields has remained
% unaltered since the moment of his enlightenment.
%
% Convinced, at all times, that no force and no sentiment could ever erase the
% infallible axioms that once guided his heart.
%
% \beat
%
% Woe unto the philosopher whose mind has been lost to the will of others.
% \stop
%

\stopcomponent
