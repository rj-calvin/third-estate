\startcomponent 01_aux_armes

\scene

\open Paris, 1791. The {\underbar H\^otel des Monnaies} salon. \sophie is seen
acting as hostess for the variety of \crowd[patrons] that gather here.
Prominently, \paine and \brissot are engaged in discussion.

\startbrissot
Ah-h, you wouldn't get it - you're an American! And you refuse to learn French!
\stop

\startpaine
I'm just saying that, clearly, the Bastille was not a strategic victory by any
measure.

I'd wager \louis thought nothing of the news when it arrived to Versailles.
\stop

\startbrissot
Ha! You should take that bet, but only because that fat pig wouldn't know a
strategic victory if he saw one.

\(beat)
And don't tell anyone I said that.
\stop

\startpaine
You must feel for the man. Anyone whipped by their wife as hard as that one
deserves compassion!
\stop

\startbrissot
Don't go around saying that.
\stop

\startpaine
Why not? I won't be speaking it in French.
\stop

\startbrissot
Don't pretend you have a monopoly on English. Gossip travels quick, and it is
often keen on abstracting away the details that absolve you.

I'd prefer that you not be thought of as belonging with fools like \marat.
\stop

\startpaine
Don't insult me! I will not be accused of associating with that bastard.
\stop

\startbrissot
You're an American, perhaps you don't understand: we French understand gossip
as a form of art.

The words of gossip compose stories woven by the beating heart of our beautiful
nation.

\(beat)
To put it simply, your technique needs work. At the moment, you are as
talentless as \marat, indeed!
\stop

\stage \brissot laughs.

\startpaine
That's nonsense. Gossip is conceit! Conspiracy! In America, we simply say what
we see, and when I see a king clearly whipped by his queen, I'll damn well say
it if I please!
\stop

\stage \sophie happens by and the remark causes her pause.

\stopcomponent
