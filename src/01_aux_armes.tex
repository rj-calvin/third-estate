\startcomponent 01_aux_armes

\scene

\open Paris, 1791. The H\^otel des Monnaies salon. \sophie is seen acting as
hostess for the variety of \crowd[patrons] that gather here. Prominently,
\paine and \brissot are engaged in discussion.

\startbrissot
Ah-h, you wouldn't get it - you're an American! And you refuse to learn French!
\stop

\startpaine
I'm just saying that, clearly, the Bastille was not a strategic victory by any
measure.

I'd wager \louis thought nothing of the news when it arrived to Versailles.
\stop

\startbrissot
Ha! You should take that bet, but only because that fat pig wouldn't know a
strategic victory if he saw one.

\(beat)
And don't tell anyone I said that.
\stop

\startpaine
You must feel for the man. Anyone whipped by their wife as hard as that one
deserves compassion!
\stop

\stage \brissot stifles a chuckle.

\startbrissot
Don't go around saying that.
\stop

\startpaine
Why not? I won't be speaking it in French.
\stop

\startbrissot
Don't pretend you have a monopoly on English. Gossip travels quick, and it is
often keen on abstracting away the details that absolve you.

I'd prefer that you not be thought of as belonging with fools like \marat.
\stop

\startpaine
Don't insult me! I will not be accused of associating with that charlatan.
\stop

\startbrissot
You're an American, perhaps you don't understand: we French understand gossip
as a form of art.

The words of gossip compose stories woven by the beating heart of our beautiful
nation.

\(beat)
To put it simply, your technique needs work. At the moment, you are as
talentless as \marat, indeed!
\stop

\stage \brissot laughs.

\startpaine
That's nonsense. Gossip is conspiracy!

In America, we simply say what we see, and when I see a king clearly whipped by
his queen, I'll damn well say it if I please!
\stop

\stage \sophie happens by and the remark causes her pause.

\startbrissot
You should heed my advice.
\stop

\startsplit

\startpaine
Advice? I'll have no advice from a hypocrite.

You accuse the king of incompetency, / yet I am decried for a joke at his
expense?
\stop

\split
\vskip\bigskipamount

\startbrissot
I asked you to not repeat that.
\stop

\stopsplit

\startbrissot
But, yes. This is true.
\stop

\startpaine
You even laughed!
\stop

\startbrissot
I laughed because I agree with you, but now is not the time to agree on such
matter.

As of now, only the most brutish and ignorant politicians accuse poor
Antoinette of treachery.

Instead, if one's goal is to form a republic, it would be more advantageous for
them to seed suspicion of the king's idiocy.
\stop

\startpaine
But queen Antoinette truly is Austrian! Clearly, she has strong connections
with France's enemies.
\stop

\startbrissot
Yes, and such simple connections appeal to the most simple of minds.

Tragic is what it is - that stupid men fixate so eagerly only on what is most
obvious.
\stop

\startpaine
Such lack of faith in your neighbors!
\stop

\startbrissot
I lack faith in politicians. My neighbors can be educated, a politician - from
my experience - cannot.

Any politician too stupid to see the truth refuses to learn it for fear of
being wrong.
\stop

\startsplit

\startpaine
If you support the republic, / wouldn't every citizen become a politician?
\stop

\split
\vskip\bigskipamount

\startbrissot
I said no such thing.
\stop

\stopsplit

\startpaine[continued]
If every citizen is responsible for electing their representation, then they
would become politicians, correct?

Suddenly, you'd no doubt lack faith in your neighbors after all!

What a sad, lonely philosopher you'd be then.
\stop

\stage The remark catches \sophie's attention.

\startbrissot
Who says I'm not a sad, lonely philosopher now?
\stop

\startpaine
Most indications from what I see.
\stop

\startsophie
What's this about a sad, lonely philosopher? Who? Where?!
\stop

\startbrissot
Yes, where is \condorcet?
\stop

\startsophie
Oh, you meant my husband! He's upstairs putting our daughter to bed.
\stop

\startbrissot
Thank you, Madame.
\stop

\instruction An infant can be heard crying in some distant room.

\stage Enter \condorcet.

\stage One of the \crowd[patrons] drunkenly mimics the infant's cry.

\stage \paine gets up to greet \condorcet, but is interrupted by \brissot.

\startbrissot
Stop! Do not mock this daughter of revolution!

Is your heart so closed to sympathy that you cannot tell that her cries are
sincere?

She cries for she has been encaged like an animal by the tyranny of her father!

She weeps alone and in turbulent anxiety; her young mind does not yet know that
tomorrow must always arrive, thus she grieves that her torment shall last an
eternity.

Oh! To weep, wondering if one's life will ever feel the joys of liberty and
freedom again - ah, my heart bleeds for her!

\(beat)
We should all be so lucky to have a voice shrill enough to cry with her.

Surely, together, we would pierce all the ears in France!
\stop

\stage \crowd[patrons] laugh.

\instruction The infant cries.

\startbrissot[continued]
You hear her, do you not? You cowards! You hear her cry but refuse to hear her
plea!

She cries for us! To come to her liberation - free her from her Bastille! To
fight for her soul and for the abolishment of bedtimes! Aux armes! Aux armes!
\stop

\startcrowd[cue=Drunkard]
Aux armes! Aux armes!
\stop

\stage \crowd[patrons] laugh.

\stopcomponent
