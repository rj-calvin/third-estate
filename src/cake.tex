\startcomponent cake

\scene

\open Fall 1791. The H\^otel des Monnaies salon. \sophie, \paine, \charlotte,
and \cabanis are waiting on \condorcet. A small cake has been prepared.

\startcharlotte
So, it's true? \condorcet really does talk to himself?
\stop

\startsophie
Every night I overhear him.
\stop

\stage \paine laughs.

\startcharlotte
What does he say?
\stop

\startsophie
It's not something I could share, really...

I shouldn't.

\(beat)
Anyhow, it's usually difficult to make any sense of it worth telling. He speaks
like he still has an imaginary friend, so you only ever hear half of the
conversation.
\stop

\startcabanis
Why do you think he does it?
\stop

\startsophie
I suspect that it's because he lacks a father figure.
\stop

\stage \charlotte shakes her head solemnly.

\startpaine
Well. That's not exactly true.
\stop

\startcharlotte
No, no, monsieur. \condorcet's father died when he was only an infant.
\stop

\startpaine
Pardon me, but that's not what I meant.
\stop

\startsophie
Oh?
\stop

\startpaine
As I've gotten to know him these last few months, it's pretty clear that he has
	{\emph some} form of father figure.
\stop

\startsophie
Oh. Yes. You're right. That's a good point.
\stop

\startcabanis
What's this?
\stop

\startpaine
It's pretty clear he thinks of \voltaire as his father.
\stop

\startsophie
I wouldn't put it that way... but... you do have a point.
\stop

\startpaine
How so?
\stop

\stage \sophie puzzles for a moment.

\startsophie
Ah! I've mistaken what distinguishes a form from a figure!

\(beat)
My original theory was that the impact of not having a father around while he
was a child changed how he behaves today.

But it's true that we can't rule out the impact other mentors can have and
shouldn't be quick to judge how these mentors compare to an authentic father.
\stop

\stage \paine puzzles on this.

\startpaine
I think that sounds about right. As far as we know, he might just find talking
to himself useful.
\stop

\startcharlotte
I think it's adorable.
\stop

\stage \sophie laughs.

\startsophie
There's a special beauty that can be learned from listening to a man like him
talk to himself.

It's a kind of beauty that's only beautiful because it's only known to me.
\stop

\stage \sophie chuckles.

\stage Enter \brissot, followed by \condorcet. \brissot initiates an applause
for \condorcet.

\startbrissot
LONG LIVE THE PHILOSOPHER!
\stop

\stage Applause. \condorcet bows.

\startbrissot[continued]
Just barely came out on top in the third round.
\stop

\startcondorcet
Ugh! It was so very exhausting...

Truthfully, if \brissot hadn't denounced the Club de la Sainte-Chapelle, I
don't think I would have won.
\stop

\startbrissot
Oh, it was no trouble. Those bastards had it coming for trying to influence the
election. But now! France can be led by science!
\stop

\stage \condorcet applauds!

\startpaine
Hold on. What's this business about the Club de la Sainte-Chapelle?
\stop

\direction \cabanis begins serving the cake.

\startbrissot
Oh. They're just royalists who, for whatever reason they fancy, don't like
people more intelligent than them see success.
\stop

\startcondorcet
They fear progress because they're ashamed of having not contributed anything
of value to it.
\stop

\stage \sophie punches \condorcet in the arm - causing him remorse.

\startcharlotte
Yes. Don't be so bitter. I agree with Monsieur \paine. I think it's awful to
denounce someone for supporting their preferred candidate.
\stop

\startpaine
I didn't say that. But, yes, I do happen to agree.
\stop

\startbrissot
What? They were trying to subvert the election using fake votes!

All to have some low-life scoundrel elected over a scholar! A respected
scholar known all across Europe!
\stop

\startcondorcet
No. I didn't think they were using fake votes.
\stop

\stage \brissot found this honesty offensive.

\startpaine
What?

\(to \brissot)
So you did denounce them based on who they supported?
\stop

\startcondorcet
Well. It was inevitable. We know this because the selected decision function
was a plurality vote: meaning ballots only provide one vote to a chosen
candidate.

You see, this decision function is incorrect for social choices since it
incentivizes voters to report their preferences strategically rather than
honestly.
\stop

\startpaine
What do you mean by "strategically?"
\stop

\startcondorcet
Oh. Uh. It means that voters can gain an advantage by inferring the way others
will be voting.
\stop

\startcabanis
How so?
\stop

\startcondorcet
Hm. Well, suppose that there are three candidates, one that you like, one that
you can tolerate, and one that you hate.

Suppose you observe that a substantial number of your peers seem to support the
candidate you hate.

That shouldn't be an issue to you because you know that you can vote honestly
to remove support from the candidate you hate.

\(beat)
However, now let's suppose you observe that the remainder of your peers seem to
support the candidate you can tolerate, but not the one that you like.

Now, we see the dilemma: if you can only choose one of the three candidates,
then should you remain honest in your selection?

Should you select the candidate you like, despite this candidate not being
likely to win?

Or do you support the candidate that you can tolerate because you worry that
the candidate you hate will win?
\stop

\startcharlotte
Ugh! Enough bitterness! \cabanis, don't encourage discussion of politics.
\stop

\startcabanis
I didn't think I was.
\stop

\startcondorcet
Oh! My apologies, \charlotte. You're right, it's been a long few days.

So, perhaps a more humorous analogy then!
\stop

\startsophie
No. I don't think...
\stop

\stage \condorcet pauses to plan out the story.

\startcondorcet
Ah! Alright!

So, suppose it is after dinner at the Tuileries, and Marie Antoinette decides
to request cake for desert.

A servant approaches the queen and informs her that they can prepare either
vanilla cake or chocolate cake.

The queen says to the servant, "I will have the vanilla cake!"

And the servant bows before departing to the royal kitchen.

\(beat)
Soon, the servant returns and informs the queen that carrot cake is also
available. "Ah!" replies the queen, "in that case, I shall have chocolate!"
\stop

\stage Laughter.

\startcharlotte
\(delighted)
Why didn't she just request chocolate to begin with?
\stop

\startcondorcet
Exactly! Her choice seems to be contingent on some third option that isn't
relevant to the first two!
\stop

\startcharlotte
Or she's just stupid!
\stop

\stage \charlotte just can't stop laughing.

\startpaine
What if there's only ever two options.
\stop

\startcondorcet
Then the phenomenon disappears - but only in appearance!

Because what you are describing, \paine[thomas], is a pairwise majority
decision, and all pairwise majority decision functions are vulnerable to
strategy.
\stop

\startpaine
I don't follow.
\stop

\startcondorcet
Suppose we stick with pairwise decisions: decisions with only two options where
the outcomes are decided by a majority vote.
\stop

\startsplit

\startpaine
Like a yay-nay --?
\stop

\split
\leavevmode

\startcondorcet
-- Yes! Precisely.
\stop

\stopsplit

\startcondorcet[continued]
In appearance these decisions seem to operate fairly, but this is only if you
are looking at just a single decision.

If instead you think about a {\emph sequence} of pairwise decisions, you will
see that the paradox returns - or, more specifically, it's revealed that the
order in which these pairwise decisions are made can be leveraged for strategic
advantages.
\stop

\startcabanis
How so?
\stop

\stage \charlotte pinches \cabanis in the arm.

\startbrissot
Oh, I could give you plenty of examples, but then we'd be discussing politics,
I'm afraid.
\stop

\startcharlotte
What I want to know is why strategy is such a big problem.
\stop

\startsplit

\startcondorcet
What?
\stop

\split

\startpaine
What?
\stop

\stopsplit

\startcharlotte
Well. You keep talking like it's a bad thing.
\stop

\startcondorcet
Of course it is! If people vote based on how they believe others will vote,
this creates intrigue and incentivizes collusion!
\stop

\stage \sophie's alarms begin going off.

\startcondorcet[continued]
It incentivizes people to denounce others; it incentivizes people to form
factions; it incentivizes people to deny the truth; it incentivizes people to
lie - to themselves, to peers, and even to their own constituencies!
\stop

\stage \sophie gets up to calm him.

\startcondorcet[continued]
If... No. It is {\emph imperative} that we discover a strategy-proof decision
function!
\stop

\startcharlotte
But isn't it a good thing to vote based on what you think is best for everyone?
\stop

\startsplit

\startsophie
Excuse me?
\stop

\split

\startcondorcet
What?
\stop

\stopsplit

\stage \condorcet considers this.

\startsophie[continued]
What do you mean "best for everyone?"
\stop

\startcharlotte
You know what I mean.
\stop

\stage Poor \cabanis is paralyzed.

\startsophie
How dare you.
\stop

\stage \condorcet stops considering it.

\startcharlotte
I'm just saying that you should consider spending less time with \olympe.
\stop

\startsophie
HOW DARE YOU! SHUT UP! SHUT UP!
\stop

\stage \condorcet tries to comfort her.

\startsophie[continued]
Don't touch me!
\stop

\stage \sophie reflexively elbows \condorcet in the stomach.

\startsophie[continued]
My god! I'm so sorry! I -- A-A-agh!
\stop

\stage Exit \sophie.

\stage \charlotte silently gets up to leave. \cabanis hesitates, then joins
her.

\stage Exit \charlotte and \cabanis.

\stage Pause.

\stage Everyone, for some reason, expects \paine to speak.

\stage He declines.

\stopcomponent
