\startcomponent god_save_the_chair

\scene

\open The Salle du Man\`ege - the makeshift auditorium that's clearly
unsuitable for a formal government. Behind the rostrum are two chairs, one
ordinary, the other is ornately gilded.

\stage Enter \condorcet, followed by \leibniz.

\startleibniz
Ugh, this is truly where the first legislative assembly of France shall meet?

Unacceptable!

The geometry is all wrong! The proportions of this hall are far too narrow and
the acoustics cannot be tolerated.
\stop

\stage Enter \voltaire.

\startvoltaire
What? Not a fan of needing to turn your neck ninety-degrees to see who's
speaking?

Or needing to cup your hands behind your ears to hear them?
\stop

\stage Enter the \deputies, including \brissot.

\stage Working-class citizens, including \robespierre, begin filling the
galleries.

\direction \robespierre is famous among the women of Paris; a group of
lower-class women accompany him in the galleries.

\stage One \deputy seems to be lost.

\startvoltaire
\(approaching the \deputy)
Aw. Look at this poor boy. The man seems completely out of his element here -
his eyes dart around the room looking for any semblance of connection to this
world.

\(to \condorcet)
Where do you suppose he should sit? To the left? Or to the right? Oh, god, how
perilous this choice must seem to him. How suffocating its consequences!
\stop

\startleibniz
Leave the man to his peace, \voltaire. This assembly is full of inexperienced
politicians and we ourselves are no different.
\stop

\startvoltaire
Nothing but greenhorns as far as a twist of the neck can tolerate.
\stop

\stage Enter \camus: the national archivist, who approaches the rostrum.

\stage \robespierre takes out his notebook.

\stage Exit \voltaire and \leibniz.

\startcamus
Good morning, gentlemen. As the national archivist, I am pleased to welcome you
to the first national legislative assembly of the kingdom of France!
\stop

\stage Applause.

\startcamus
As demanded by our newly ratified constitution, it shall be made understood
that following my departure from the rostrum, no man or deputy may be permitted
to speak at the podium without the explicit permission of the presiding
assembly president and may only be done so during active sessions that are to
take place daily from the hours of ten o'clock to sunset.

\(beat)
By law, and to ensure the impartial initiation of these proceedings, the first
president shall not be selected by vote, as no man yet has the authority to
call for one, but rather by the selection of the eldest sitting deputy.
\stop

\stage \condorcet shakes his head.

\startcamus
May, uh, Monsieur Batault rise to the president's chair.
\stop

\stage The elderly \president emerges from the crowd and, with effort, climbs
to the ordinary chair.

\startdeputy[number=1]
DESPICABLE that our president is given such a shameful chair!
\stop

\startsplit

\startdeputy[number=2]
HAVE YOU NO RESPECT FOR DECORUM, HEATHEN?
\stop

\split

\startdeputy[number=3]
LONG LIVE THE NATION!
\stop

\stopsplit

\startdeputy[number=4]
LONG LIVE THE KING!
\stop

\stage The deputies begin shouting at each other.

\stage \camus is confused. He gestures as if to intervene but hesitates and
looks to the \president.

\stage The \president has no idea what to do. \camus spots a bell and hands it
to the \president, who begins ringing it.

\stage No effect.

\startdeputy[number=1]
\(charging to the rostrum)
I REQUEST THE FLOOR!
\stop

\stage \camus eagerly evades the conflict.

\stage Exit \camus.

\startdeputy[number=1]
By law, the king of France is no more sacred than that of the public.

Any symbol to the contrary is an insult to the good citizens of France!

It is imperative that this gilded chair be destroyed immediately and replaced
with a chair that is identical to the president's.
\stop

\startsplit

\startrightminority
LONG LIVE THE KING!
\stop

\split

\startleftminority
LONG LIVE THE NATION!
\stop

\stopsplit

\stage Shouting ensues as the \deputy descends from the rostrum.

\startdeputy[number=2]
\(charging to the rostrum)
HOW DARE YOU! I REQUEST THE FLOOR!
\stop

\stage The \president knows he's supposed to do something...

\startdeputy[number=2]
How dare you insult our honorable sovereign. You, monsieur, have spat on the
sacred history of our kingdom! Have you no shame?
\stop

\startsplit

\startleftminority
DESTROY THE CHAIR! DESTROY THE CHAIR!
\stop

\split

\startrightminority
GOD SAVE THE CHAIR! GOD SAVE THE CHAIR!
\stop

\stopsplit

\startdeputy[number=2]
GOD SAVE THE CHAIR!
\stop

\stage More shouting. The \deputy descends from the rostrum.

\direction \condorcet begins to tune out the assembly.

% TODO: conclude the scene.

\stage Enter \voltaire, followed by \leibniz.

\startvoltaire
What a conundrum, professor. What say you on the matter? I myself am utterly
torn.
\stop

\startleibniz
Surely, you must be joking.
\stop

\startvoltaire
I never joke about politics.
\stop

\startleibniz
You always joke about politics.
\stop

\startvoltaire
I had nothing to do with that. In politics, the jokes always write themselves!
\stop

\stage \voltaire laughs.

\stopcomponent
