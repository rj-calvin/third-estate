\startcomponent acts_of_the_apostles

\scene

\open The H\^otel des Monnaies salon. \condorcet, \paine, and \sophie are
in a discussion.

\startpaine
Hmm... I disagree.
\stop

\startcondorcet
You disagree? How so?
\stop

\startpaine
I don't believe that the rights of man should be subject to a charter or
decree.

The act of creating, or formulating, our human rights as written articles
implies that they are vulnerable to revocation by a majority at the expense of
the minority.
\stop

\startsophie
So you prefer to think of rights as being innate to nature? Such as with
Hobbes or Locke?
\stop

\startpaine
I wouldn't know. I don't read other people's opinions - I prefer to think for
myself.
\stop

\stage \condorcet finds this hilarious. \paine and \sophie chuckle along.

\startpaine[continued]
But yes, I think the rights of man are inexorably linked to our physical bodies
in relation to the physical world.
\stop

\startcondorcet
I agree with your principles, but I don't think this excludes us from
formulating natural rights using language.

In fact! This observation of rights being inherit to the physical world should
compel us to study them in the same manner as our sciences!
\stop

\startpaine
I don't see how that could be possible.
\stop

\startcondorcet
Consider that the utility of having rights established in writing provides us
with two important properties:

first, this allows us to properly inform people of their rights -

just because the rights are natural does not mean that we are born with an
awareness of them -

and second, it allows us to identify when these rights are being contradicted
by some other law or action.
\stop

\startsophie
I don't see how that would enable us to study our rights scientifically.
\stop

\startcondorcet
Oh. I suppose it doesn't...
\stop

\startpaine
Though, you do bring up an interesting --
\stop

\startcondorcet
-- Ah, perhaps if we consider the application of \leibniz's laws of
contradiction and sufficient reason!

\(beat)
Wait... no, perhaps not... Hmm...
\stop

\stage \paine chuckles.

\stage Enter \brissot carrying a stack of papers.

\startpaine
Ah, \brissot!
\stop

\startbrissot
The news is in. \condorcet is officially a politician.
\stop

\startcondorcet
Did you bring me an official certificate?
\stop

\stage \condorcet cackles.

\startsophie
Wait, what do you bring? Are those...
\stop

\startbrissot
The latest broadsheets.
\stop

\startsophie
No! I forbid this! My husband has no need to subject himself to slander!
\stop

\startbrissot
You're husband is a politician now, Madame.

It's imperative - now, more than ever - that he be informed in how he the
public perceives him.
\stop

\startsophie
NO!
\stop

\startcondorcet
It's no worry, \sophie! \brissot has a point.
\stop

\startsophie
No, he doesn't, he just thinks --
\stop

\startbrissot
-- The man can speak for himself.
\stop

\startsophie
How DARE you!
\stop

\startcondorcet
\sophie. It's okay. Truly. \brissot is right - it can't hurt to be informed.
\stop

\stage \sophie sneers.

\startbrissot
Let's begin with Le Babillard, \lafayette reads this one.
\stop

\startpaine
How do you know that?
\stop

\startbrissot
It's my job to know what people read.
\stop

\startcondorcet
\(reading from the paper)
"\condorcet, a conniving geometer who can only measure his own filth and who
has never achieved anything but villainy, has embarrassed himself by betraying
the monarchy."

\(beat)
It's true, I'm certainly no Euler!

Though, I doubt a man like \lafayette could ever suspect me of amounting to
villainy.
\stop

\stage \condorcet cackles.

\startbrissot
This one is a favorite of \barnave.
\stop

\startcondorcet
\(reading from the paper)
"\condorcet betrays his allies and his closest friends in his gratuitous and
hypocritical call for republicanism."

\(beat)
Ah... yes. I thought it might hurt some feelings...
\stop

\startpaine
I say don't let it stop you from saying what you think is right.
\stop

\startcondorcet
Exactly!
\stop

\startbrissot
\barnave himself has a quote in that issue.
\stop

\startcondorcet
Oh?

\(scans the paper)
Ah. "\condorcet, a man who has inexplicably been awarded numerous academic
titles through no merit of his own,

he advocates for a republic despite the knowledge that by the very same
passion that a nation can overthrow royalty, it could just as well be used to
overthrow a republic."

Did \barnave even read my speech? I had it published!
\stop

\startbrissot
Probably not.
\stop

\startcondorcet
\(continues reading)
"It seems that recent events have drawn in a few learned men in geometry who
are clearly ill-suited for applying their science to politics.

We have drawn them in, I say, by this talk of abstractions, but we all know
that true patriots are drawn in only by realities."
\stop

\startpaine
Now {\underbar that's} how you do slander!
\stop

\stage \brissot laughs.

\startcondorcet
You were right, \brissot. It may not be pleasant, but it's useful to know.
\stop

\startbrissot
Ah! And the best for last.
\stop

\startsophie
Is that...?
\stop

\startbrissot
Yep, the Acts of the Apostles.
\stop

\startsophie
Who on earth reads that?
\stop

\startbrissot
You would be surprised.
\stop

\startcondorcet
Oh, this one is in verse!

\(reading from the paper)
\vbox{"Seeing the effigies}
\vbox{Of the filthy Rousseau,}
\vbox{Of the impious Voltaire,}
\vbox{Of the monster Mirabeau,}

\vbox{Joseph said to Jesus: what a profane display!}
\vbox{Of unbelievers, of scoundrels,}
\vbox{Here are the busts: but, alas!}
\vbox{Your image is absent:}

\vbox{Then came into the room}
\vbox{A notorious cockroach,}
\vbox{The latest disciple of atheism,}
\vbox{Shifty eyes, dirty beard,}

\vbox{What is this, said Joseph, this frightful expression?}
\vbox{This is the \condorcet[marquis],}
\vbox{The philosopher draped in gold,}
\vbox{The first cousin of satan."}
\stop

\stage \condorcet chuckles.

\startsophie
What's the point of this? Did you really just come here to sour our day?

It's not even good verse.
\stop

\startcondorcet
My day hasn't soured! Today I learned that I'm a shifty, conniving cockroach
with a fantastical beard!
\stop

\stage \condorcet mimics a cockroach. \sophie resists the urge to be humored by
it.

\startpaine
You'd think Jesus would take offense to Joseph's petty name calling.
\stop

\startbrissot
That certainly doesn't stop our holy apostles here.

\(to \sophie)
If it makes you feel better, I feature on the next page.
\stop

\startcondorcet
\(reading from the paper)
\vbox{"Appeared he whose soul}
\vbox{Is the color of the skin}
\vbox{Of those for whom his baroque brain is inflamed," --}

-- Oh, how crude!
\stop

\startbrissot
I get it all the time.

To my mind, it doesn't even amount to an insult.

Certainly, not in the way they think it does, at least.
\stop

\startcondorcet
\vbox{"His name forms an insult, a disgusting wound:}
\vbox{In London he was a spy,}
\vbox{Here he is still a rogue:}
\vbox{Ah, it's Brissot, I bet."}

\(tossing the paper aside)
Enough of that.
\stop

\stage \condorcet picks up one of the other papers.

\startpaine
I didn't know you were a spy.
\stop

\startbrissot
Spy... journalist... depends on how threatened they feel by what I do.
\stop

\stage \paine laughs.

\startcondorcet
Oh! Did you see this?
\stop

\startbrissot
I bet you mean the petition at the Champ-de-mars.
\stop

\startsophie
A petition?
\stop

\startcondorcet
There's going to be public petition at the Champ-de-mars demanding the new
constitution become a republic!
\stop

\startbrissot
Technically it doesn't say republic.

But yes, that's what it amounts to.
\stop

\startsophie
How wonderful! We should all attend!
\stop

\startbrissot
I mean, yes, you should... Though...
\stop

\startcondorcet
Though?
\stop

\startbrissot
Well... I'm not sure...

\(beat)
At the \crowd[jacobins] this morning, \robespierre presented an argument that's
been turning in my head all day.
\stop

\startsophie
Oh, \robespierre? Many women speak very highly of him. Ever since the march on
Versailles.
\stop

\startbrissot
Yeah, I've noticed.
\stop

\startpaine
Who is \robespierre?
\stop

\startbrissot
That's just the thing. I'm not sure that I know.

\(beat)
He argued today that a public gathering like this would provide the royalists
with an opportunity to silence all talk on republicanism through a show of
force.
\stop

\startcondorcet
What? A peaceful petition would be an opportunity for a demonstration of force?
\stop

\startbrissot
Exactly my thought.
\stop

\startsophie
You don't believe him, though.
\stop

\startbrissot
You didn't hear him speak.

\(beat)
His argument was razor sharp, expertly devised, sinister in its implications.

He made it sound like such an outcome was a tactical certainty.
\stop

\startsophie
But you don't believe him.
\stop

\startbrissot
I don't.
\stop

\startcondorcet
Then what has you worried?
\stop

\startbrissot
Him.
\stop

\stopcomponent
