\startcomponent aux_armes

\scene

\open The H\^otel des Monnaies salon. In the midst of the variety of \patrons
are \paine and \brissot engaged in a conversation.

\startbrissot
Ah-h, you wouldn't get it - you're an American! And you refuse to learn French!
\stop

\startpaine
I'm just saying that, militarily, capturing the Bastille had little strategic
value.
\stop

\startbrissot
Strategic value! Strategic value? What's this value have anything to do with
the strength of our hearts and minds?
\stop

\startpaine
If it's only hearts and minds, a pamphlet would have sufficed.
\stop

\startbrissot
Gah! A pamphlet? You think we haven't thought of trying that?

Oh, god, if only it were that easy when your slaver is the one who chooses the
price of bread.
\stop

\startpaine
Your slaver? You'd compare yourself to the poor lads in Saint-Domingue?
\stop

\startbrissot
Is that not the language to use for anyone who is not free?

Perhaps, to you, freedom means the right to go about your day,
business-as-usual.

No. We fight here in Paris for our very way of life - for the natural rights
that must be pried from the iron claws of an ancient regime!
\stop

\startpaine
But who are you fighting? \louis has abdicated his throne. All you need to do
is declare a republic.
\stop

\startbrissot
Gah! If only it were that easy...
\stop

\startpaine
Mm, a republic isn't exactly common sense here in Paris, is it?
\stop

\startbrissot
Common sense? Merely the act of speaking the word could have you arrested.

\beat

The National Assembly is very eager to "make an example" of anyone who might
wish to "subvert the constitution."
\stop

\startpaine
What constitution? The National Assembly wrote the whole thing for a king who
shattered his own crown just a few days ago.
\stop

\stage \brissot laughs.

\startbrissot
Oh. Didn't you hear?

Our idiot-king was found in Varennes. In fact, he should be arriving back to
the Tuileries later tonight.

You could go and see him if you don't mind being crushed under the weight of
his shame.
\stop

\startpaine
I don't see how this changes anything. This was clearly an act of treason.
\stop

\startbrissot
Oh, indeed it was. Absolutely {\emph indisputable} proof that all the
conspiracies of him and Antoinette colluding with Austria were true.

\beat

Oh, and let's not forgot the damned {\emph confession} the moron left for us!

Yes, the letter that that puts in no uncertain terms how he truly feels about
the revolution.

How he swore his oath to the rights of man through gritted teeth and bitter
hatred in his heart.

If only this was enough to eliminate people's prejudices!
\stop

\startpaine
To be fair, I'd imagine that's how any monarch would feel about a revolution.
\stop

\startbrissot
It's undeniable evidence that \louis does not act in the interest of the
nation.

He acts only to protect his status among the other monarchs of Europe.

Yet, truth be told, this isn't what angers me about his treachery.

\beat

Would you like to know the thing I hate most about him?
\stop

\stage \paine raises his brow.

\startbrissot
His incompetency. How clumsy this whole charade is.

How shameless and cowardly he acts.
\stop

\stage Enter \sophie.

\startbrissot
I hate it. I hate it more than you can know.

Because it means that our true enemy... is our own idleness.
\stop

\startsophie
\condorcet is putting our daughter to sleep, he'll be down shortly.
\stop

\startbrissot
\beat{deep breath}

Excellent news, madame.
\stop

\stage \sophie sits with the two as a peer.

\startpaine
So, how is it that you know \condorcet?
\stop

\startsophie
He's my husband, silly!
\stop

\startbrissot
Oh, how I know him?

He and I met through the society of friends of the blacks, an association
that I formed a few years before the Estates-General.
\stop

\startpaine
Much like the, uh --
\stop

\startbrissot
-- Like the one in London, yes! I was very inspired during my time there.

\beat{hushed}

Did you know that it's custom for slavers in Saint-Domingue to gouge out the
eyes of their house slaves so that they can churn butter "without
distractions?"
\stop

\startsplit

\startsophie
Ugh! Don't bring up such morbid --
\stop

\split
\vskip\bigskipamount

\startpaine
-- As a matter of fact, I've heard that about American slavers.
\stop

\stopsplit

\startpaine
I don't make my mind up based on rumors. My disgust for slavery is justified by
simple reasoning.
\stop

\startbrissot
Ha! \condorcet keeps trying to tell me the same thing.
\stop

\startpaine
You don't agree that it's simple reasoning?
\stop

\startbrissot
I don't agree that reason is how you gain support for a cause.
\stop

\startpaine
Then how do you go about it?
\stop

\startbrissot
Showmanship.
\stop

\startsophie
I'm sure charlatans like \marat would agree with you.
\stop

\startbrissot
\marat is a maniac who appeals to the most savage prejudices in men.

What he agrees or doesn't agree with means nothing to me anymore. Nor should it
to anyone!

\beat

Proper showmanship only requires proper knowledge of one's audience.

So, yes, it's true: \marat is a capable showman~- and that's because he's
honest and a crowd seeks honest men.

But honesty is ugly: it pollutes the world with archaic rituals and revolting
prejudices.

And his crowd is as filthy as the heart that feeds on them.

\beat{gestures}

I, however... cultivate a finer audience.
\stop

\startpaine
Is that why you partnered with a philosopher?
\stop

\startbrissot
Partner might be putting it strongly. \condorcet is a friend.
\stop

\startpaine
Do the two of you disagree often?
\stop

\startbrissot
No. But I see him as a liability in politics.
\stop

\startsophie
A liability? Excuse me?
\stop

\startbrissot
Madame, you must admit that your husband is a brilliant man, but a terrible
politician.
\stop

\stage \sophie is conflicted.

\startpaine
Don't be rude to our lovely hostess, \brissot.
\stop

\startbrissot
You're right, \paine[thomas]. I apologize for my forwardness, madame, but I
must also ask that you don't mistake my criticism for hatred.

I consider my relationship with \condorcet to be deeply valuable.

\beat{to \paine}

The truth is, of all the honest men in Paris, his honesty is the only one that
keeps me sane.
\stop

\direction An infant can be heard crying from the other room.

\stage Enter \condorcet.

\startdrunkard
Wa-a-ah! Wa-a-a-a-ah!
\stop

\stage A few \patrons laugh. \sophie sneers at the \drunkard.

\stage \brissot stands to make a scene.

\startbrissot
Who's this insufferable bastard?

You dare call yourself a patriot?

Is your heart so closed to sympathy that you cannot tell that this child's
cries are sincere?

\beat{chuckles}

She cries because she begs for the freedom that's been taken from her~-
entrapped, then abandoned by her own father, no less!

Oh, and such a tragedy it is that this little patriot does not yet know that
tomorrow must always arrive, thus she grieves that her torment shall last for
eternity.
\stop

\stage The \patrons laugh.

\direction The infant cries louder.

\startbrissot
Oh! To weep alone and anxious, wondering if one's life will ever feel the joys
of liberty again - ah, my heart bleeds for her!

We should all be so lucky to have a voice shrill enough to cry with her.

Surely, together, we would pierce all the ears in France!
\stop

\stage The \patrons laugh.

\startbrissot
You laugh, yet you hear her, do you not? You cowards!

You hear her cry but refuse to hear her plea!

She cries for us! To come to her liberation - free her from her Bastille! To
fight for her soul and the abolishment of bedtimes! TO ARMS! TO ARMS!
\stop

\startdrunkard
To arms! To arms!
\stop

\stage Laughter from the \patrons, followed by applause.

\stage \sophie punches \brissot in the arm.

\stage Exit \sophie.

\startcondorcet
ALAS, my villainy has been exposed by the vigilant journalism from \brissot!
\stop

\stage \brissot laughs.

\direction The infant quiets down.

\startbrissot
Thank you, friend, thank you. Your honesty always reveals the deepest
kindnesses.

\beat

Now, let us join our new friend from America! Mister \paine[full].
\stop

\startpaine
Pleasure to make your acquaintance.
\stop

\stage \paine hands \condorcet a letter. \condorcet doesn't pay much mind to
it.

\startcondorcet
You've been to Paris before, haven't you?
\stop

\startbrissot
I am certain you two will get along.
\stop

\startcondorcet
Oh, will I be talking his ear off?
\stop

\startbrissot
Assuming he's as polite as I take him for.
\stop

\stage \condorcet seems delighted!

\startpaine
\beat{clears throat}

Yes, I was the one who joined Colonel Laurens in seventeen eighty-one when we
were raising funds for the war with Britain.
\stop

\startcondorcet
A shame we hadn't the pleasure of meeting at the time.

And I can't tell you how heartbroken Paris had been upon hearing of Franklin's
passing - myself, first and foremost among them.
\stop

\startpaine
I didn't hear the news until my landing in London last year.

In fact, I learned that he had passed not long after writing my letter of
introduction.
\stop

\stage \condorcet, with a gasp, suddenly realizes the sanctity of the letter he
holds in his hands.

\startpaine
\beat{to \brissot}

You're right. He and I are going to get along just fine.
\stop

\stage \condorcet finds the letter very moving.

\startcondorcet
Oh, well met, Mister \paine. Well met!

\beat{to \brissot}

An American patriot and philosopher has arrived to France just in time to help
us follow in their footsteps! Wouldn't you agree?
\stop

\startbrissot
Heaven knows France has {\emph plenty} of patriots, and you alone are already a
philosopher worth a thousand.

\beat{to \paine}

Honestly, I think you'd be better off enjoying our excellent wine and women
than to get involved in our politics!
\stop

\startpaine
That's what I did last time; do you also plan to gift me another six million
livres for the pleasure?
\stop

\stage \brissot laughs. \condorcet furrows his brow as he puts on a grin.

\stage Enter \sophie.

\startcondorcet
\sophie! Oh, \sophie!
\stop

\stage \sophie seems defensive. \condorcet hands her the letter of
introduction.

\startcondorcet
Mister \paine has brought us a letter from Franklin!
\stop

\startsophie
Oh, this is from Franklin?
\stop

\stage \sophie begins reading.

\startcondorcet
It was the last letter that he ever wrote...
\stop

\startpaine
I... doubt that's true... More than likely it would be his will and testament.
\stop

\startsophie
Hm. Well... well, he certainly writes favorably of Mister \paine, I suppose.
\stop

\startpaine
I'm flattered, but the letter was only a matter of~--
\stop

\startcondorcet
\beat{to \sophie}

-- Are you disappointed?
\stop

\startsophie
Oh, no, I don't wish to be impolite!
\stop

\startcondorcet
Oh, hush! We are all enlightened people, so there should be no fear to discuss
honestly with one another regarding the phenomena of the mind.

\beat{to \brissot and \paine}

Wouldn't you agree?
\stop

\stage \brissot gestures as if to object~--

\startsophie
-- You're right. Indeed, I don't quite know how to label the feeling, so it
would help to discuss it.
\stop

\stage \condorcet is delighted!

\startsophie
Well, I never knew Franklin.

I never had the pleasure of meeting him in person.

\beat

Of course, I respect Franklin based on the eloquence of his writing, but
clearly this letter means something special to you because you knew him as a
personal friend, \condorcet[c].

\beat

For me, however, the letter simply evoked the self-same feelings that I've
always had for him.

So, I believe that I felt pressured to feel as passionate about the letter
as you seemed to be.
\stop

\startcondorcet
Perhaps, if I understand correctly, that you felt like I was attempting to
modify your relationship with Franklin?
\stop

\startsophie
Yes! It felt as if you insisted that I should feel the same way that you do,

despite my relationship with Franklin only being with his writing.

Yet, I still felt protective because I cherish all relationships that I form
with others~--
\stop

\startsplit

\startsophie
-- no matter the form they take!
\stop

\split

\startcondorcet
-- no matter the form they take!
\stop

\stopsplit

\startcondorcet
Eureka!
\stop

\startpaine
Why does the form of the relationship matter in the first place?
\stop

\startsophie
Well. I believe that the relationships we form through writing are different
than the relationships we form through speaking.
\stop

\startpaine
I don't feel that way.
\stop

\stage Pause.

\startsophie
Well. Why not?
\stop

\startpaine
Because words have the same meaning whether they are heard or read, do they
not?
\stop

\stage \condorcet begins puzzling on this.

\startsophie
No. No. I don't think so.
\stop

\startcondorcet
Ah! I think I might know!

\beat

Perhaps the difference is because relationships formed in person are
interactive, while those formed through writing are not.
\stop

\stage \sophie gasps!

\startsophie
YES! That's it!
\stop

\startpaine
But you've failed to refute my rebuttal.
\stop

\startcondorcet
Hm? What rebuttal?
\stop

\startbrissot
Perhaps you've also forgotten why we are here.
\stop

\startcondorcet
Hm? What? Oh! Yes!

\beat

So, \paine[thomas]... What do you think France should do with her king?
\stop

\startsplit

\startbrissot
\beat{disapproving}

\condorcet, you really need to learn how to~--
\stop

\split
\leavevmode

\startpaine
-- Exile.
\stop

\stopsplit

\startbrissot
Exile? Unacceptable! He would immediately go to Austria and unite the
emigrants!
\stop

\startpaine
The who?
\stop

\startcondorcet
The people who fled from the revolution to Austria.
\stop

\startpaine
Oh, like what \louis is doing now?
\stop

\stage \condorcet and \sophie laugh.

\startbrissot
What he attempted to do.
\stop

\startpaine
And again, I ask: so what? Let them have their king.

He's a terrible leader and a worse tactician.
\stop

\stage \brissot laughs.

\startpaine
I say, just come out and call for a republic.
\stop

\startcondorcet
Not while people remain uneducated.
\stop

\startpaine
How do you mean?
\stop

\startcondorcet
The people don't yet know what a republic even is!

Many in France are not even able to read, let alone understand how to count a
majority!
\stop

\startpaine
Well, Paris seems quite literate.

When I arrived, I mistook all the pamphlets as a sign of an early autumn.
\stop

\startcondorcet
Well, we live in a country larger than~--
\stop

\startpaine
-- My apologies - yes, of course.
\stop

\startbrissot
No. You're right, \paine. Whoever controls Paris controls the country.
\stop

\startpaine
I thought this was about education?
\stop

\startbrissot
This is about education.

\beat

But making sure that enlightenment is made universal is the grand vision! Not
the first step!
\stop

\stage \paine chuckles.

\startcondorcet
All that matters now is that we begin educating the people to ensure that the
public reason is prepared to begin participating in elections.
\stop

\startpaine
I disagree.
\stop

\startcondorcet
What!?
\stop

\startpaine
You say that's {\emph all} that matters. I disagree.

I believe it is only {\emph some} of the matter.
\stop

\startcondorcet
Ah! You're right! Thank you.

\beat

However, it is nonetheless IMPERATIVE!
\stop

\startpaine
Of course.
\stop

\startcondorcet
Though, you do make a good point, \brissot.

Perhaps we should consider first educating the people of Paris who already know
how to read since that shall be more efficient given our resources.
\stop

\startbrissot
Hear, hear!
\stop

\startsophie
So, a journal for the virtues of a new republic! Yes!
\stop

\startcondorcet
An excellent idea!

\beat

And, I say, we shouldn't show any fear in choosing its name.

We should proudly announce to the world who we all know in our hearts to be!

We shall call it, the Republican!
\stop

\startbrissot
\condorcet, you understand that there's real discussion happening in the
National Assembly on whether to begin using martial law to stop any mention of
republicanism in Paris.
\stop

\startsplit

\startcondorcet
What!?
\stop

\split

\startsophie
Nonsense! That is clearly a violation of the freedom of the press!
\stop

\stopsplit

\startcondorcet
Precisely!
\stop

\startpaine
Well, actually, would that not fall under the freedom of speech?
\stop

\startcondorcet
\beat{delighted}
Ah! A good question! For you see, it's both!

The freedom of the press {\emph implies} the freedom of speech since the
construction of alphabets derives from the association of signs to the sounds
produced by our voices.

\beat

You see? It generalizes!
\stop

\startpaine
Ah! I see. That's very clever, \condorcet.
\stop

\stage \brissot is stumped.

\startsophie
However, is there not still a distinction between the rights as they pertain to
an individual as opposed to a coalition?
\stop

\stage \condorcet gasps!

\startcondorcet
An excellent observation!

\beat

Yet, I might argue that we do not need to worry {\emph how} a message is formed
for it to have the right to exist as the product of human labor.

And I can observe that this is equivalent to the freedom of the press in
itself.
\stop

\startpaine
Then why not extend this right further? Why limit this to the presses?

Would you not say that all men are the owners of their labor, and protection
from its infringement be the freedom preserved by the law?
\stop

\startcondorcet
I limit myself to the press because not all products of human labor are a
benefit to social welfare.

For example, people should not profit from the pillage of war, the enslavement
of humans, or the distribution of opium.
\stop

\startpaine
As grim as those things may be, I fail to see how that doesn't contradict your
earlier claim that all products of human labor have the right to exist?
\stop

\stage \condorcet puzzles for a moment.

\startcondorcet
Hm. You're right. My language on this needs to be made more precise.

It is important that I communicate this clearly since I may need this for my
speech to the \jacobins tomorrow.
\stop

\stage \brissot shakes his head.

\startbrissot
The \jacobins prefer {\emph concision}, not precision, \condorcet.
\stop

\startpaine
\beat{to \brissot}

Speaking of: what are the sentiments of the \jacobins on this matter about the
king?
\stop

\startbrissot
Oh, I don't know.

\robespierre hasn't told them yet!
\stop

\stage \brissot laughs.

\startsophie
Well, what do you believe, then?
\stop

\startbrissot
Are you asking what I believe will happen, or what I believe should happen?
\stop

\startcondorcet
Well, what you think {\emph should} happen, of course!

We aren't here to act like prophets!
\stop

\stage \condorcet cackles. \sophie is humored by it.

\stage \brissot suddenly seems perceptive of his surroundings.

\startbrissot
If you pledge that you won't share this information with others?
\stop

\stage Pause. All nod incredulously.

\startbrissot
The truth is, I wish for a fate worse than death for \louis.

\beat{beat}

Not because of what he has done, but for what he is and what he represents to
the royalists.
\stop

\startcondorcet
No.
\stop

\stage Pause.

\startcondorcet
No; no; no, no, no.
\stop

\stage Pause. The sound of troubled breathing.

\startcondorcet
Ah. I see. I think, I see. Wait...

\beat

NO!

\beat{stands proudly}

It is not necessary that we kill \louis[king]!

Humans shape their tools which in turn shape themselves, so we know that the
tools we shape for our collaboration are also tools that shape our behavior.

\beat

Meaning that by selecting the correct tools, we can select the correct
behaviors.

Therefore, once a republic is established with a sound constitution, and once
all citizens are educated with the ability to read and interpret this
constitution,

people will then begin to transition their faith away from superstitions and
priests and accept that our reason alone is sufficient to ensure the prosperity
of the human species!
\stop

\stage Pause. Silence.

\startsophie
My love, I had trouble following your reasoning. You spoke a bit too quickly.

\beat

But I agree with your conclusion.
\stop

\startbrissot
I'm vetoing all of that from your speech.
\stop

\stopcomponent
