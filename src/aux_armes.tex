\startcomponent aux_armes

\scene

\open The H\^otel des Monnaies salon. In the midst of the variety of \patrons
that gather here are \paine and \brissot.

\startbrissot
Ah-h, you wouldn't get it - you're an American! And you refuse to learn French!
\stop

\stage Enter \sophie.

\startpaine
I'm just saying that, clearly, the Bastille was not a strategic victory by any
measure.
\stop

\startsophie
My husband should be down in a moment to meet you gentlemen.
\stop

\startbrissot
Ah, thank you, Madame.
\stop

\stage \sophie disengages.

\startpaine
You know, I'd wager \louis thought nothing of the news when it arrived to
Versailles.

Probably more confused by it than anything.
\stop

\startbrissot
Ha! You should take that bet, but only because that fat pig wouldn't know a
strategic victory if he saw one.

\(beat)
And don't tell anyone I said that.
\stop

\startpaine
You must feel for the man. Anyone whipped by their wife as hard as that one
deserves compassion!
\stop

\stage \brissot stifles a chuckle.

\startbrissot
Don't go around saying that.
\stop

\startpaine
Why not? I won't be speaking it in French.
\stop

\startbrissot
Don't pretend you have a monopoly on English. Gossip travels quick, and it is
often keen on abstracting away the details that absolve you.

I'd prefer that you not be thought of as belonging with fools like \marat.
\stop

\startpaine
Don't insult me! I will not be accused of associating with that charlatan.
\stop

\startbrissot
Perhaps you don't understand: we French understand gossip as a form of art.

The words of gossip compose stories woven by the beating heart of our beautiful
nation.

\(beat)
To put it simply, your technique needs work. At the moment, you are as
talentless as \marat, indeed!
\stop

\stage \brissot laughs.

\startpaine
That's nonsense. Gossip is conspiracy!

In America, we simply say what we see, and when I see a king clearly whipped by
his queen, I'll damn well say it if I please!
\stop

\startbrissot
You should heed my advice.
\stop

\startsplit

\startpaine
Advice? I'll have no advice from a hypocrite. You accuse the king of
incompetency, / yet I am decried for a joke at his expense?
\stop

\split
\vskip\bigskipamount

\startbrissot
I asked you to not repeat that.
\stop

\stopsplit

\startbrissot
But, yes. This is true.
\stop

\startpaine
You even laughed!
\stop

\startbrissot
I laughed because I agreed with you, but now is not the time to agree on such
matter.

As of now, only the most brutish and ignorant politicians accuse poor
Antoinette of treachery.

Instead, if one's goal is to form a republic, it would be more advantageous for
them to seed suspicion of the king's idiocy.
\stop

\startpaine
But queen Antoinette surely {\underbar is} Austrian. Clearly, she has strong
associations with France's enemies.
\stop

\startbrissot
Yes, and such simple associations appeal to the most simple of minds.

Tragic is what it is - that stupid men fixate so eagerly only on what is most
obvious.
\stop

\startpaine
Such lack of faith in your neighbors!
\stop

\startbrissot
I lack faith in politicians. My neighbors can be educated, a politician - from
my experience - cannot.

Any politician too stupid to see the truth refuses to learn it for fear of
being wrong.
\stop

\startpaine
Perhaps that's why they fixate on the obvious?
\stop

\startbrissot
Perhaps.
\stop

\startsplit

\startpaine
Though, consider: if your goal is to form a republic, / wouldn't every citizen
become a politician?
\stop

\split
\vskip\bigskipamount

\startbrissot
I said no such thing.
\stop

\stopsplit

\startpaine[continued]
In a nation where every citizen is responsible for electing their
representation, citizens would ostensibly become politicians, correct?

Thus, in a republic, you'd no doubt lack faith in your neighbors after all!

What a sad, lonely philosopher you'd be then.
\stop

\startbrissot
Who says I'm not a sad, lonely philosopher now?
\stop

\startpaine
Most indications from what I see.

\(beat)
So, how is it that you know \condorcet?
\stop

\startbrissot
He and I are outspoken abolitionists. We bonded through our cause a couple of
years ago.
\stop

\startpaine
Encouraging! Though, how difficult is the situation in France? How much
influence do the sugar colonies have in the assembly?
\stop

\startbrissot
% TODO: verify.
Shockingly little. Insultingly little.
\stop

\startpaine
Oh, I see...

\(beat)
Though, it must be an asset to have a philosopher as a colleague.
\stop

\startbrissot
It would hurt his feelings to say this, but \condorcet may happen to be the
least talented politician I've ever met.

Far worse than you, if it makes you feel better.
\stop

\startpaine
You'd slander your ally?
\stop

\startbrissot
If you consider honesty to be slander.

If by honesty, you think of filth like \marat.

Or the crude such as yourself - oh, pardon me!
\stop

\stage \paine is humored.

\startbrissot[continued]
I keep \condorcet close because his honesty keeps me sane.
\stop

\direction An infant can be heard crying in some distant room.

\stage Enter \condorcet.

\stage One of the \patrons drunkenly mimics the infant's cry.

\startbrissot
Stop! Do not mock this daughter of revolution!

Is your heart so closed to sympathy that you cannot tell that her cries are
sincere?

She cries for she has been encaged like an animal by the tyranny of her father!

Her young mind does not yet know that tomorrow must always arrive, thus she
grieves that her torment shall last for eternity.

\(beat)
Oh! To weep alone and anxious, wondering if one's life will ever feel the joys
of liberty and freedom again - ah, my heart bleeds for her!

We should all be so lucky to have a voice shrill enough to cry with her.

Surely, together, we would pierce all the ears in France!
\stop

\stage The \patrons laugh.

\direction The infant cries.

\startbrissot[continued]
You hear her, do you not? You cowards! You hear her cry but refuse to hear her
plea!

She cries for us! To come to her liberation - free her from her Bastille! To
fight for her soul and the abolishment of bedtimes! To arms! To arms!
\stop

\startdrunkard
To arms! To arms!
\stop

\stage The \patrons laugh.

\stage \condorcet claps.

\startcondorcet
Doubtless that my villainy would have been exposed by anyone but the talented
journalist, \brissot.
\stop

\startbrissot
Thank you, friend, thank you.
\stop

\startcondorcet
\(to \paine)
And friends, indeed. I sense that I recognize you.
\stop

\stage \paine procures a letter.

\startbrissot
This friend is from America. \paine[formal].
\stop

\startpaine
I bring a letter of introduction.
\stop

\startcondorcet
From Jefferson?

\(inspects letter)
% NOTE: this letter is fiction as far as I am aware.
% Franklin did have correspondence with \condorcet, and Franklin did likely
% write letters of introduction for \paine, but I don't know what this letter
% would have looked like.
Oh, Franklin! My word, he had passed just last year! \sophie! It's a letter
from Franklin!
\stop

\startsophie
Truly!?
\stop

\startcondorcet
Truly!

\(to \paine)
May I keep this?
\stop

\startpaine
% CONSIDER: if this is realistic for a letter of introduction.
It's addressed to you.
\stop

\startcondorcet
Ah. Yes - yes, of course.
\stop

\stage Brushing past \paine, \condorcet ignores the offered handshake to begin
reading.

\stage \brissot smiles to \paine and gestures that they sit with \condorcet.

\stage Pause.

\startcondorcet[continued]
Ah!

\(shakes \paine's hand)
Pleasure, mister \paine! But you have been to Paris before, correct? Weren't
you with... oh, what's his name?
\stop

\startpaine
Colonel Laurens.
\stop

\startcondorcet
Ah, yes. Colonel Laurens, that's right.
\stop

\stage \sophie joins the group and begins reading the letter.

\startcondorcet[continued]
Certainly, you should no longer recognize the France you knew! Indeed, you have
arrived at such an extraordinary time!
\stop

\startpaine
I had heard that \louis fled Paris.
\stop

\startcondorcet
You are present to witness the formation of the first republic of France! How
exciting!
\stop

\startbrissot
The assembly remains firmly monarchist.

\stop

\startcondorcet
Of course they're monarchists, \brissot! They have lived their whole lives
under monarchy.

Surely, therefore, we must expect that they require education on the merits of
republican ideals.
\stop

\startsplit

\startcondorcet[continued]
Once we address the \jacobins\ / and explain the reasoning of the republican
system, I have no doubt that they shall see the folly of supporting a king who
hates them.
\stop

\split
\vskip\bigskipamount

\startbrissot
I still insist you reconsider.
\stop

\stopsplit

\startpaine
Hates them?
\stop

\startbrissot
Heh, he left a letter on his desk before he fled - quite the tell-all.

Apparently, the oath he swore before god and nation last year was done through
seething teeth.

\(beat)
Honestly, I cannot fathom the logic that compelled that man to make such an
embarrassing maneuver.

How is it possible for someone to live their whole life surrounded by politics
and not learn a single thing about it?
\stop

\startsophie
The assembly was holding the royal family essentially hostage.

He probably just wanted to ensure his family's safety.
\stop

\startbrissot
Nonsense! He was given identical accommodations to his life in Versailles.
\stop

\startsophie
You jest! Ever since the royals arrived in Paris they have been living in
constant fear of the public.
\stop

\startbrissot
The Tuileries is furnished with plenty of guards.
\stop

\startsophie
Oh, that bleeding of yours! You conflate their privileges with their consent!
\stop

\startpaine
Agreed.
\stop

\startsplit

\startbrissot
\paine[informal], I am betrayed!
\stop

\split

\startsophie
Thank you.
\stop

\stopsplit

\stopcomponent
