\startcomponent on_the_republic

\scene

\open Jacobin club. \robespierre is sitting alone, meditating.

\stage Enter the various \crowd[jacobins].

\stage As the \crowd[jacobins] notice \robespierre, they all rush to confront
him.

\startcrowd[cue={Jacobins}]
\robespierre! \robespierre!
\stop

\startanonymous[cue={Jacobin 1}]
\robespierre! Are you here to learn of the republic?

With the king having fled, do you support his deposition?

Do you support the republic?
\stop

\startanonymous[cue={Jacobin 2}]
All of Paris is desperate to know! Do you support the republic?
\stop

\stage The \crowd[jacobins] fall to a dead silence.

\stage \robespierre, without moving, without acknowledgement, produces a sly
grin.

\startrobespierre
What even is a republic?
\stop

\stage The \crowd[jacobins] erupt into deliberation.

\stage Enter \brissot and \paine.

\stage Enter \antonelle.

\stage \antonelle takes to the podium and rings a loud bell.

\startantonelle
Gentlemen! Gentlemen! Silence!
\stop

\stage The \crowd[jacobins] settle down.

\startantonelle[continued]
Tonight, we are privileged to have a distinguished guest.

\(beat)
We are honored to have the \condorcet[formal] join us to speak on his support
for the formation of a new French republic.
\stop

\stage The \crowd[jacobins] become rowdy. \antonelle rings the bell again.

\startantonelle[continued]
Calm! Calm everyone!

\(beat)
Monsieur de \condorcet is a reputed mathematician and scientist at the Royal
Academy of Sciences and has served as its permanent secretary for the last
fourteen years.

His work has focused on the innovative application of mathematics to the study
of men, their commerce, and their politics.
\stop

\stage Quiet speculation is heard from the \crowd[jacobins] on what this could
mean.

\startantonelle[continued]
\condorcet is a disciple of the late statesman, Turgot; the late mathematician,
d'Alembert; and even our dearly missed \voltaire.

\(beat)
With that, I hope you all welcome our new guest and friend.
\stop

\stage \antonelle steps away. The \crowd[jacobins] applaud.

\stage Enter \condorcet.

\stage \condorcet takes to the podium, produces his written speech from his
pocket, and clears his throat.

\stage He is an atrocious orator.

\startcondorcet
The citizens of France, having demonstrated their courage in claiming their
freedom and having now demonstrated their calm firmness in contemplating the
danger that has just threatened it, no longer require eloquence in the call to
liberty.

It is therefore to their reason alone that we must speak on the security of a
peaceful future that is worthy of an enlightened people.

\(beat)
In the face of recent events, the question that we are all obliged to ask
ourselves is this: is a king necessary for the establishment of liberty and
the protection from tyranny?

\(beat)
No, gentlemen. It is not.

\(beat)
The friends of royalty would argue that the power of an executive that is
established and limited by written law is far less formidable than that of a
usurped power that has no other limits than those of skill and audacity.

\(beat)
However, these people neglect to consider that a government formed from a sound
and enlightened constitution would see the powers of the army, the finances,
and the courts be adequately divided and distributed between disparate
departments and deputies.

Therefore, it would be necessary to have destroyed, corrupted, or denatured all
these powers simultaneously before being able to aspire to tyranny.

And since any enlightened constitution would guarantee the freedom of the
press, this threat is therefore impossible for a government where deputies are
subject to frequent elections.

\(beat)
Consider that any popular tyrant can only act under a mask that hides their
hypocrisy.

Thus, the power ensured by the press to force these aspirants to walk with
their faces uncovered shall render them all harmless.

Let us therefore not allow ourselves to harm our own prosperity and happiness
out of fear for an imaginary danger.
\stop

\stopcomponent
