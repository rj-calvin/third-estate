\startcomponent on_the_republic

\scene

\open Jacobin club. \robespierre is sitting alone, meditating. His eyes are
always obscured by a pair of tinted spectacles.

\stage Enter the various \jacobins.

\stage As the \jacobins notice \robespierre, they all rush to confront him.

\startjacobins
\robespierre! \robespierre!
\stop

\startjacobin[number=1]
\robespierre! Are you here to argue for the republic?

With the king having fled, do you support his deposition?

Do you support the republic?
\stop

\startjacobin[number=2]
All of Paris is in suspense! We must know if you support the republic!
\stop

\stage The \jacobins fall to a dead silence.

\stage \robespierre, without moving, without acknowledgement, produces a sly
grin.

\startrobespierre
What even is a republic?
\stop

\stage The \jacobins erupt into deliberation.

\stage Enter \brissot and \paine, followed shortly by the \president.

\stage The \president takes to the podium and rings a loud bell.

\startpresident
Gentlemen! Gentlemen! Silence!
\stop

\stage The \jacobins settle down.

\startbrissot
Monsieur \president, I request the floor!
\stop

\startpresident
\brissot! The floor is yours!
\stop

\stage \brissot takes to the podium. A healthy number of \jacobins applaud.

\startbrissot
\stop

\startcondorcet
Monsieur \president, I would like to request the floor.
\stop

\startpresident
The philosopher wishes to speak!
\stop

\stage A healthy number of \jacobins applaud.

\startpresident[continued]
\condorcet, the floor is yours.
\stop

\stage \condorcet takes to the podium. A handful of \jacobins applaud.

\stage He is an atrocious orator.

\startcondorcet
T-the citizens of - of France, having demonstrated their courage in claiming
their freedom and having now demonstrated their calm firmness in - in
contemplating the danger that has just threatened it, no longer require
eloquence in the call to liberty.

It is therefore to their reason alone that we must speak on the security of a
peaceful future that is worthy of an enlightened people.

\(beat)
In the face of recent events, the question that we are all obliged to ask
ourselves is this: is a king necessary for the establishment of liberty and
the protection from tyranny?

\(beat)
No, gentlemen, it is not.

\(beat)
The friends of royalty would argue that the power of an executive that is
established and limited by written law is far less formidable than that of a
usurped power that has no other limits than those of skill and audacity.

However, these people neglect to consider that a republic formed from a sound
and enlightened constitution would see the powers of the army, the finances,
and the courts be adequately divided and distributed between disparate
departments and deputies.

Therefore, it would be necessary to adequately divided and distribut(ed) --
oh, I read this part already.

\(clears throat)
Uhm... t-therefore, it would be necessary to have destroyed, corrupted, or
denatured all these powers simultaneously before one could aspire to tyranny.

And since any enlightened constitution would guarantee the freedom of the
press, this threat is therefore impossible for a republic where deputies are
subject to frequent elections.

\(beat)
Consider that any popular tyrant can only act under a mask that hides their
hypocrisy.

Thus, the power ensured by the press to force these ambitious people to walk
with their faces uncovered shall render them all h-harmless.

Let us therefore not allow ourselves to harm the progress of the human species
and the happiness of our nation out of fear for an imaginary danger.
\stop

\stage Polite applause from the \jacobins. \brissot and \robespierre do not
contribute.

\stage \condorcet seems pleased.

\startcondorcet[continued]
No longer do we live in a time when laws can only be upheld through divine
intervention.

No longer are we prejudiced by the impious superstitions that suppose that any
man can speak on behalf of a supreme intelligence so long as he wears the
proper attire.

\(beat)
Yet, those friends of royalty, those who remember the events of history, but
who do not know history, remain frightened by the corruption of ancient
republics.

But let them examine these republics, and they will always see a division
between those who are sovereigns and those who are subjects.

Then they will understand the true forces that corrupt the privileged and
seduce the impoverished, and it will be clear that these forces cannot exist in
a nation where equality is made universal to all people who inhabit it - no
matter their race, their wealth... or - or their gender.
\stop

\stage The \jacobins are confused by this.

\startcondorcet[continued]
It is therefore not the tyranny of an individual that we must fear, but the
most odious and sinister tyrannies that exist when one multitude rules over
another through abuses of law and finance.

\(beat)
But is such abuse what the friends of liberty demand? Those who only accept
reason as the sole master of their fate?

I ask you: at whose expense could we satisfy the greed of our leaders?

What conquered provinces could a French general strip to buy our votes?

Would such an ambitious man propose to us, as he had to the Athenians, to levy
tributes on our neighbors in order to fund new temples and festivals?

Will he promise our soldiers, as he had to the Romans, the pillage of Spain or
Syria?

No, doubtless!

\(beat)
I ask you: how could we feed the greed of the few in a nation that refuses to
exploit their allies?

In a nation that refuses the conquests of war?

In a nation that refuses to be subjugated and built by citizens who refuse to
subjugate others?

\(beat)
I answer you, my friends, that it is because we refuse to be a kingly people
that we will remain a free people.
\stop

\stage \condorcet folds up his speech.

\stage Applause from the \jacobins builds from trepidation to enthusiasm.

\stage \brissot and \robespierre do not contribute.

\stopcomponent
