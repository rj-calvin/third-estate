\startcomponent chorus

\scene

\open \condorcet is writing at his desk.

\stage Enter \voltaire.

\startvoltaire
Republic, republic! What even is a republic?
\stop

\stage Enter \leibniz.

\startleibniz
Ah! The answer is simple! A republic is a category of monad!
\stop

\startvoltaire
Oh, \leibniz, whatever shall we do with you?
\stop

\startleibniz
Do you suppose France is ready for it?
\stop

\startvoltaire
Oh, dear god, keep it in your pants, professor.

Nobody wants to see your monads!
\stop

\startleibniz
Ehem, a republic, \voltaire. Do you think France is ready for a republic?
\stop

\startvoltaire
Aha. Well, first, we'll have to see what France does with their king.
\stop

\startleibniz
For when they abolish the monarchy? I suppose this is true.
\stop

\direction The \stranger whispers to \condorcet.

\stage \condorcet begins to think.

\startcondorcet
But what should be done about the royalists? How do we refute their prejudices?
\stop

\startvoltaire
Ah, of course. The age old question: without a divine monarch, how do we
prevent the rise of tyranny?

\(beat)
That's a tough one. A real stinker.
\stop

\startleibniz
Ah, well! The trouble with republics is that they heavily depend on having a
proper decision function.

Particularly, a decision function that does not incentivize strategy.

\(beat)
You see, a decision function is a method of aggregating the preferences of an
assembly into a single preference representing the whole.

Allow me to illustrate!
\stop

\stage \leibniz wheels in a blackboard.

\startdrawing

\startformula
\hskip-\leftedgewidth
\startmathalignment[n=3]
\NC \text{First Estate} \NC :\; \NC A > B > C \NR
\NC \text{Second Estate} \NC :\; \NC B > C > A \NR
\NC \text{Third Estate} \NC :\; \NC C > B > A \NR
\stopmathalignment
\stopformula

\stopdrawing

\startleibniz
Suppose we have three estates that are tasked with deciding between three
policies: A, B, and C.

Policy A represents the status quo and is the preferred choice of the
clericalism of the first estate.

Policy B represents the moderate change that is preferred by the nobility of
the second estate.

Lastly, policy C represents a radical change and is preferred by the
disadvantaged third estate.

\(beat)
An example of a decision function would be pairwise majority.

Here, we see that two of three prefer policy B over policy A, two of three
prefer C over A, and two of three prefer B over C.

Thus, this decision function would aggregate these preferences into the result:
\stop

\stage \leibniz appends:

\blank[line]

\startdrawing
\startMPcode
path p; p := fullcircle xscaled 3cm yscaled 1cm;
draw p;
label(btex $B > C > A$ etex, center p);
\stopMPcode
\stopdrawing

\startcondorcet
Ah, I see that it's identical to the second estate's judgement.
\stop

\startleibniz
Yes, you observe the flaw with pairwise majority decisions.

You see, why would the third estate submit their preferences honestly when they
know that this will end up favoring the second estate?

Instead, the third estate can strategically change their preference as so:
\stop

\stage \leibniz edits:

\startdrawing

\startformula
\hskip-\leftedgewidth
\startmathalignment[n=3]
\NC \text{Third Estate} \NC :\; \NC C > A > B \NR
\stopmathalignment
\stopformula

\leavevmode
\hskip-\leftedgewidth
\startMPcode
path p; p := fullcircle xscaled 3cm yscaled 1cm;
draw p;
label(btex ? etex, center p);
\stopMPcode
\stopdrawing

\startleibniz[continued]
Now, two of three prefer A to B, two of three prefer B to C, and two of three
prefer C to A.
\stop

\startcondorcet
It forms a cycle!
\stop

\startleibniz
Correct. By acting strategically, the third estate can sabotage majority
decisions.
\stop

\startcondorcet
Now, what would be an example of a decision function that is strategy-proof?
\stop

\startsplit

\startleibniz
I don't know of one yet!
\stop

\split

\startvoltaire
Oh, I can think of one!
\stop

\stopsplit

\startcondorcet
You know of one, \voltaire?
\stop

\startvoltaire
Indeed.

\(beat)
Suppose our decision function simply selects the first estate every time.

Then the process is strategy-proof since nobody needs to strategize.

The first estate does not since it knows that it's preferences are always going
to succeed.

And the second and third estates do not since nothing they can do matters!
\stop

\startleibniz
Admittedly, this is correct.

But that would be a trivial decision function.

We are only interested in non-trivial ones.
\stop

\startvoltaire
It seems pretty damn non-trivial to me.
\stop

\startcondorcet
Have faith! Such a non-trivial decision function is necessary for the
advancement of progress!
\stop

\startleibniz
One step forward, to a destination infinitely far away!
\stop

\stage Enter \sophie, knocking at the door.

\direction \leibniz and \voltaire disappear.

\startsophie
My love. Please keep this in your head. You keep waking the baby.
\stop

\startcondorcet
Oh! My apologies!
\stop

\stage Exit \sophie.

\stage \condorcet returns to his desk and stares at his writing.

\stage Pause.

\startcondorcet
Boy, that wasn't very helpful at all...
\stop

\stopcomponent
