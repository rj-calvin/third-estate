\startcomponent election

\scene

\open Fall 1791. The H\^otel des Monnaies salon. \sophie, \paine, \charlotte,
and \cabanis are waiting on \condorcet. \sophie is carrying an infant. A small
cake has been prepared.

\startcharlotte
So, it's true? He does talk to himself?
\stop

\startsophie
Every night I overhear him. There's a certain beauty in what can be learned
from overhearing someone speak to themselves.

It's like listening to proof that there exists a vast, colorful world outside
of your own.

It's language without anything to gain and without anything to lose.

It's honest, and it's sincere.
\stop

\startcabanis
But why does he do it?
\stop

\startsophie
I suspect that it's because he lacks a father figure.

His father had died while he was only an infant.
\stop

\startcabanis
Ah, that would explain a few things.
\stop

\stage Enter \condorcet.

\startcabanis[continued]
And speaking of!
\stop

\stage Encouraging applause.

\startcrowd
Congratulations!
\stop

\startcondorcet
Oh.
\stop

\startsophie
Look \eliza! Your father is now a legislator!
\stop

\startcabanis
By a meager four votes. Congratulations, my friend.
\stop

\startcondorcet
Oh, I expected as much. The constitution has cursed us with the burden of
plurality voting.
\stop

\stage \paine laughs.

\stage \paine shakes \condorcet's hand and respectfully nods.

\stage \charlotte gives \condorcet a hug.

\startcharlotte
We are so delighted to hear of your success!

Come! Come! My sister has brought us a cake!
\stop

\stage \condorcet awkwardly obliges.

\startcondorcet
Oh, oh, she has?
\stop

\startcharlotte
\(to \sophie)
So very shy for a deputy.
\stop

\startcondorcet
You all shouldn't have gone through the trouble for me.
\stop

\stage \cabanis laughs.

\startsophie
Trouble? Oh, no. No, you misunderstand.

Our celebration is not for you.
\stop

\startcondorcet
How do you mean?
\stop

\startsophie
Celebration is made for the celebrators, not for the celebrated.

We join together to rejoice not in the love that is given, but in the love that
is shared.
\stop

\stage \condorcet is touched.

\startsophie[continued]
Now, eat the cake, it was expensive.
\stop

\stage \condorcet nods and partakes in the cake.

\startcabanis
So what's this about - eh, what - plurality voting, you say?
\stop

\stage \charlotte slaps his arm.

\startcharlotte
\cabanis! We're here to celebrate. It's no place to bring up politics.
\stop

\startcabanis
Dear, we are literally celebrating the man getting elected to the assembly.
\stop

\startcharlotte
You know what I mean.
\stop

\startcabanis
That wasn't even my question!
\stop

\startsophie
Enough.
\stop

\startcondorcet
No need to worry, \charlotte.

Indeed, it's actually a wonderful question! I'd be delighted to speak on it.
\stop

\startcabanis
See?
\stop

\startsophie
Enough!
\stop

\startcondorcet
In the science of decision making, plurality voting is a form of election where
preferences are aggregated exclusively by the top preference of each
participant.

One ballot, one vote.
\stop

\startpaine
Mm.
\stop

\startcondorcet
Ah, yes. You would doubtless recognize this, \paine[informal].
\stop

\startcabanis
What's so wrong with it?
\stop

\startcondorcet
Plurality elections incentivize voters to lie about their preferences in order
to gain a strategic advantage in the results.

Consider that in a single-winner system, where each voter is given only one
vote, there is no incentive to vote for any candidate that is unlikely to win,
even if such a candidate is the most preferred choice by the voter.

As a result, the voters do not submit their ballot based on what they truly
prefer, but rather based on what they believe other voters will prefer.

Therefore, plurality decisions are not strategy-proof, and thus, their outcomes
cannot reliably represent the preferences of their constituency.
\stop

\startcabanis
What's an example of a method that's strategy-proof?
\stop

\startcondorcet
I don't know of one yet!
\stop

\startcabanis
Oh...
\stop

\startpaine
Why not just make it a rule that ballots are kept a secret?
\stop

\startcondorcet
You would need to make it a rule to prevent people from talking about their
preferences altogether, which would contradict the right of free speech.
\stop

\startpaine
Huh.
\stop

\startcharlotte
I don't understand.
\stop

\startcondorcet
Oh, please! I love questions!
\stop

\startcharlotte
You use the word "strategy" like it's a bad thing, but isn't it a good thing to
vote based on what you think is best for others?
\stop

\startcondorcet
What an insightful perspective!

Well, you see, another way to see the issue with strategy is to think about it
as causing unexpected or irrational outcomes.

Allow me to demonstrate with a story.
\stop

\startcharlotte
Oo! A story! You were right, \cabanis; this was an excellent choice of topic!
\stop

\startcondorcet
So, suppose it is after dinner at the Tuileries, and Marie Antionette is
ordering desert.
\stop

\startcharlotte
Ordering cake, no doubt!
\stop

\startcondorcet
Ordering cake, indeed! A servant approaches the queen and informs her that they
can prepare either vanilla cake or chocolate cake.

The queen says to the servant, "I will have the vanilla cake."

The servant bows and proceeds to the kitchen.

But after a moment, the servant returns and informs the queen that carrot cake
is also available.

\(beat)
"Ah," replies the queen, "in that case, I shall have chocolate!"
\stop

\stage Laughter.

\startcharlotte
I suppose the queen is up to some secret scheme involving her choice of cake!
\stop

\startcondorcet
Precisely!
\stop

\startcharlotte
So! \condorcet! What do you plan to do now that you're a deputy?
\stop

\startsplit

\startcabanis
I thought we weren't bringing up politics?
\stop

\split
\leavevmode

\startcondorcet
Oh. I am going to lead the committee of public instruction!
\stop

\stopsplit

\startcondorcet[continued]
I believe that before any progress is possible, that it is imperative that we
build a strong education system that's free for everyone!
\stop

\startpaine
To teach what? And by who?
\stop

\startcondorcet
Well, I'm still drafting the details of my proposal.
\stop

% TODO: address the pacing in this sequence.
\stage Enter \brissot\ - something occupies him.

\startbrissot
My apologies for being tardy.

Congratulations on the election, friend!

We'll have it praised in tomorrow's press!

\(beat)
Though, uh, may I speak with you? I need to speak with someone I can trust.
\stop

\startcondorcet
Oh! Well, it's a, uh --
\stop

\startbrissot
-- It's about \robespierre.
\stop

\startcabanis
Who is \robespierre?
\stop

\startcharlotte
He's the charming man who wants to reduce the price of bread.
\stop

\startbrissot
Everyone wants to reduce the price of bread.
\stop

\startsophie
Enough!
\stop

\stage Pause.

\startcharlotte
Oh, well, uh, I -- you're right, \sophie. Perhaps this talk isn't appropriate.
\stop

\startbrissot
\(to \condorcet)
\robespierre is back out from hiding after the massacre.
\stop

\startcondorcet
What about \marat?
\stop

\startbrissot
He and \danton\ --
\stop

\startcharlotte
-- Well, I understand how important you men treat your politics.

\sophie, let's get \eliza to bed.
\stop

\startsophie
What? There's no reason we can't participate.
\stop

\startcharlotte
\sophie, don't be so stubborn.
\stop

\startsophie
I'm being {\underbar stubborn}?
\stop

\startcabanis
\charlotte! Maybe it's a good time to head home? While the sun's still up?
\stop

\stage Pause.

\startcharlotte
Yes, thank you all so much for your hospitality.
\stop

\startcabanis
Absolutely, everything was great.
\stop

\stage Exit \charlotte and \cabanis.

\stage \sophie slaps \brissot.

\startsophie
Damn you!
\stop

\stage Exit \sophie.

\startbrissot
My... uh, my apologies, \condorcet.

I don't know what came over me.
\stop

\startpaine
Maybe you'll compensate for the cake.
\stop

\startbrissot
At these prices? Maybe not.
\stop

\startpaine
Maybe \robespierre would have a solution to that?
\stop

\stage \paine is humored, \brissot is not.

\startbrissot
\robespierre is up to something. He's dividing the \jacobins, forcing out
\barnave and the other royalists.
\stop

\startpaine
I thought you hated them?
\stop

\startbrissot
I do. But the point of democracy is to have healthy debate, not eliminate your
allies!

\(beat)
But more importantly, by forcing out the royalists, \robespierre gains
substantial leverage over the remaining \jacobins.
\stop

\startpaine
What do you know about \robespierre?
\stop

\startbrissot
That's just the thing...

That's what worries me most about him.

\(beat)
I have no idea who he is. Who is \robespierre?

I ask what he reads, and he reads everything.

I ask what he believes, and he believes in the general will.

I ask what he wants, and he wants virtue.

I ask who he fucks, and he fucks nobody!

The snake lives like a damn priest!

\(beat)
But I believe I may have discovered his weakness.

I believe that \robespierre's tactics are leveraging a new kind of prejudice.
\stop

\startcondorcet
A new kind of prejudice?
\stop

\startbrissot
Yes. It's a prejudice that doesn't depend on physical traits.
\stop

\startcondorcet
Really? That's a fascinating idea, \brissot. What do you call it?
\stop

\startbrissot
The prejudice of suspicion.
\stop

\startpaine
What makes suspicion a prejudice?
\stop

\startbrissot
Because it's just as dangerous as prejudice.

In fact, it may be the most dangerous prejudice of all.

You see, anyone can be accused of being suspicious since it doesn't depend on a
person's physical traits, and yet it's impossible to refute it!

How can someone suspicious convince others that they are sincere when that is
exactly the thing a suspicious person would do?
\stop

\startcondorcet
Ah, this problem will be fixed when we discover a non-trivial, strategy-proof
decision function.
\stop

\startpaine
What makes a decision function non-trivial?
\stop

\startbrissot
No. This... this is serious. This problem is real.

\robespierre is a villain - I think that this prejudice of suspicion is capable
of destroying the revolution.

\(beat)
\robespierre. \robespierre, needs to be stopped.

And for that we need to be the ones responsible for abolishing the monarchy.

\(beat)
\condorcet. I need your support in declaring war with Austria.
\stop

\startsplit

\startcondorcet
WHAT?
\stop

\split

\startpaine
What?
\stop

\stopsplit

\startbrissot
Listen, friend. I know you have your misgivings about it, but we need to lay on
the offensive against \louis before the radicals do.

We need to make sure that we can maintain a majority when we abolish the
monarchy so that we're in control when we write the next constitution.
\stop

\startcondorcet
Yes, it's imperative that we adopt a new constitution that allows for
amendments.

\(beat)
But - but war is against my principles.

So, I'm sorry \brissot, but I will have to decline.
\stop

\startbrissot
This isn't a matter of principle. It's necessary.
\stop

\startcondorcet
That hasn't been proved yet.
\stop

\stage \brissot takes a deep breath.

\startbrissot
I understand. I understand.
\stop

\startcondorcet
I apologize, \brissot.
\stop

\startbrissot
No need to. I respect your conviction. Patriots are allowed to have different
opinions, after all.

\(beat)
But I should be the one apologizing.

I didn't mean for my rudeness to sour your celebration.
\stop

\startcondorcet
Oh, that's okay, I don't think --
\stop

\startbrissot
-- All of France should be celebrating the philosopher joining us at the new
Legislative Assembly!

It's one step forward toward the end of tyranny and prejudice!
\stop

\stopcomponent
