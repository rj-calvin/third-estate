\startcomponent election

\scene

\open Fall 1791. The H\^otel des Monnaies salon. \sophie, \paine, \charlotte,
and \cabanis are waiting on \condorcet. \sophie is carrying an infant. A small
cake has been prepared.

\startsophie
Do you hear him, \eliza? Is that your father entering?
\stop

\stage Enter \condorcet.

\stage Encouraging applause.

\startcrowd
Congratulations!
\stop

\startcondorcet
Has it happened!?
\stop

\startsophie
It has! Look \eliza! Your father is now a legislator!
\stop

\startcabanis
By a meager four votes. Congratulations, my friend.
\stop

\startcondorcet
Oh, I expected as much. The constitution has cursed us with the burden of
plurality voting.
\stop

\stage \paine laughs.

\stage \paine shakes \condorcet's hand and respectfully nods.

\stage \charlotte gives \condorcet a hug.

\startcharlotte
We are so delighted to hear of your success!

Come! Come! My sister has brought us a cake!
\stop

\stage \condorcet awkwardly obliges.

\startcondorcet
Oh, oh, she has?
\stop

\startcharlotte
\(to \sophie)
So very shy for a legislator.
\stop

\startcondorcet
You all shouldn't have gone through the trouble for me.
\stop

\stage \cabanis laughs.

\startsophie
Trouble? Oh, no. No, you misunderstand.

Our celebration is not for you.
\stop

\startcondorcet
How do you mean?
\stop

\startsophie
Celebration is made for the celebrators, not for the celebrated.

We join together to rejoice not in the love that is given, but in the love that
is shared.
\stop

\stage \condorcet is touched.

\startsophie[continued]
Now, eat the cake, it was expensive.
\stop

\stage \condorcet nods and partakes in the cake.

\startcabanis
So what's this about - eh, what - plurality voting, you say?
\stop

\stage \charlotte slaps his arm.

\startcharlotte
\cabanis! We're here to celebrate. It's no place to bring up politics.
\stop

\startcabanis
Dear, we are literally celebrating the man getting elected as a legislator.
\stop

\startcharlotte
You know what I mean.
\stop

\startcabanis
That wasn't even my question!
\stop

\startsophie
Enough.
\stop

\startcondorcet
No need to worry, \charlotte.

Indeed, it's actually a wonderful question! I'd be delighted to speak on it.
\stop

\startcabanis
See?
\stop

\startsophie
Enough!
\stop

\startcondorcet
In the science of decision making, plurality voting is a form of election where
preferences are aggregated exclusively by the top preference of each
participant.

One ballot, one vote.
\stop

\startpaine
Mm.
\stop

\startcondorcet
Ah, yes. You would doubtless recognize this, \paine[informal].
\stop

\startcabanis
What's so wrong with it?
\stop

\startcondorcet
Plurality elections incentivise voters to lie about their preferences in order
to gain a strategic advantage in the results.

Consider that in a single-winner system, where each voter is given only one
vote, there is no incentive to vote for any candidate that is unlikely to win,
even if such a candidate is the most preferred choice by the voter.

In other words, the voters do not submit their ballot based on what they truly
prefer, but rather based on what they believe other voters will prefer.

Therefore, plurality decisions are not strategy-proof, and thus, their outcomes
cannot reliably represent the preferences of their constituency.
\stop

\startcabanis
What's an example of a method that's strategy-proof?
\stop

\startcondorcet
I don't know yet.
\stop

\startcabanis
What do you mean?
\stop

\startcondorcet
I've spent several years trying to figure out a proper method for voting that
does not incentivize strategy.

So far, everything I have come up with has some kind of flaw or exploit that
can't be avoided.
\stop

\startpaine
Why not just make it a rule that ballots are kept a secret?
\stop

\startcondorcet
You would need to make it a rule to prevent people from talking about their
preferences altogether, which would contradict the right of free speech.
\stop

\startpaine
Huh.
\stop

\startcharlotte
I don't understand.
\stop

\startcondorcet
Oh, please! I love questions!
\stop

\startcharlotte
You use the word "strategy" like it's a bad thing, but isn't it a good thing to
vote based on what you think is best for others?
\stop

\startcondorcet
What an insightful perspective!

Well, you see, another way to see the issue with strategy is to think about it
as causing unexpected or irrational outcomes.

Allow me to demonstrate with a story.
\stop

\startcharlotte
Oo! A story! You were right, \cabanis; this was an excellent choice of topic!
\stop

\startcondorcet
So, suppose it is after dinner at the palace of Versailles, and Marie
Antionette is ordering desert.
\stop

\startcharlotte
Ordering cake, no doubt!
\stop

\startcondorcet
Ordering cake, indeed! A servant approaches the queen and informs her that they
can prepare either vanilla cake or chocolate cake.

The queen says to the servant, "I will have the vanilla cake."

The servant bows and proceeds to the kitchen.

But after a moment, the servant returns and informs the queen that carrot cake
is also available.

\(beat)
"Ah," replies the queen, "in that case, I shall have chocolate!"
\stop

\stage All laugh together.

\startcharlotte
I suppose the queen is up to some secret scheme involving her choice of cake!
\stop

\startcondorcet
Precisely!
\stop

\stage Enter \brissot.

\stage Something occupies him.

\startbrissot
How... have you heard yet about whether...

\(beat)
Oh, I see. You won, then?
\stop

\startcondorcet
I'm officially a legislator!
\stop

\startbrissot
Wow. Yes, that's wonderful. I'll... uh, I'll have your praises published in
tomorrows paper.
\stop

\stage \brissot paces nervously.

\startsophie
\brissot, you seem bothered.
\stop

\startbrissot
Yes. I am. Terribly. I haven't slept in thirty-two hours.
\stop

\startcabanis
Good god, man! Do you need a prescription? I can see about stopping by my
office.
\stop

\startbrissot
No. I need to talk to people I can trust. Who are you two?
\stop

\startsophie
This is my sister, \charlotte, and her husband, \cabanis. But~--
\stop

\startbrissot
-- \robespierre. \robespierre has made a move on the \crowd[jacobins].
\stop

\startcharlotte
Oh, I, uh, I don't know if this is really appropriate.
\stop

\startbrissot
I've never seen anything like it! It was... it was... extraordinary!

It was perhaps the most brilliant political maneuver I have ever seen!
\stop

\startcabanis
Who is \robespierre?
\stop

\startcharlotte
He's the charming man who wants to fix the price of bread.
\stop

\startbrissot
Everyone wants to fix the price of bread.
\stop

\startcabanis
Alright, so what's his deal then?
\stop

\startbrissot
I... I don't know... I don't know who he is.

I've gone around to whoever I could corner asking for information on the guy,
but nobody knows a damn thing!

I ask what he reads, and he reads everything.

I ask what he desires, and he desires virtue.

I ask what he believes, and he believes in the general will.

I ask who he fucks, and he fucks nobody!
\stop

\startsophie
Enough! We are here to celebrate my husband's success!
\stop

\startbrissot
Right, right. My - my apologies.
\stop

\startcondorcet
I... well, yes, it's true - this is meant to be a loving occasion.

But I've never seen you like this before, \brissot.

\(to \sophie)
I think he really needs our help.

\(beat)
\brissot, let us help you, but first let's make sure we are all comfortable.

\charlotte, I understand you may not feel right with this, so perhaps \brissot
and I can discuss in another room?
\stop

\startcharlotte
I... I understand that you men care so very much about this talk, so I think
\sophie and I should... uh...
\stop

\startsophie
No, we have just the same right to be here as anyone!
\stop

\startcharlotte
Oh, \sophie[personal], you and your stubbornness. Come. Let's... let's get
\eliza to bed.
\stop

\stage A fire is lit in \sophie.

\startcabanis
Please, don't bring this back up again, \charlotte.

Here, it's a fine time to begin heading back to the house - while the sun's
still up.
\stop

\stage \charlotte nods.

\stage Exit \cabanis and \charlotte.

\stage \sophie slaps \brissot.

\startsophie
Damn you!
\stop

\stage \sophie storms off to an adjacent room.

\startbrissot
I... I'm sorry, \condorcet. I... I just...

\(beat)
Thank you.
\stop

\startcondorcet
Maybe you can pay for the cake?
\stop

\startbrissot
At these prices? I'd rather lend you my left arm.
\stop

\stage \brissot gives a dry chuckle.

\startbrissot[continued]
You're right. That's a fine idea. I'll write a letter and... make sure my
article about your election is one for the ages.

\(beat)
The scholar and philosopher blesses our new legislature, after all.
\stop

\startpaine
Must have been pretty rough at the \crowd[jacobins].
\stop

\startbrissot
\marat and his soldiers came by.
\stop

\startcondorcet
Oh, no...
\stop

\startbrissot
Yep. They were giving us an ultimatum: help them kill the assembly, or be
killed with them.

\(beat)
And then suddenly... the quiet, unassuming, self-righteous \robespierre stands
up and does the impossible.

\(beat)
He sides with them.

\(beat)
He wins them over by confessing how \barnave had approached him with a bribe in
exchange for supporting martial law...
\stop

\startcondorcet
What? \barnave really did that?
\stop

\startbrissot
Of course not! HE LIED! It was genius!

The man stood on that tribune, stared death in the eye... and bluffed.

\(dry laugh)
Now, he's won the heart of the \crowd[sansculottes], and with their support, he
now also has the \crowd[jacobins] by the throat.
\stop

\startpaine
What are you going to do?
\stop

\startbrissot
I don't know.
\stop

\startcondorcet
\brissot.
\stop

\startbrissot
What?
\stop

\startcondorcet
You're worried over nothing.

You and I are legislators now. \robespierre is not. And neither are the
\crowd[jacobins], nor the \crowd[sansculottes].
\stop

\startbrissot
Right... Right. You're right. Yes. Yes... \louis is our only real opponent.

Defeating him is all that needs to be done.

With him out of the picture, none of them have any leg to stand on.
\stop

\startpaine
What the hell are you planning, \brissot?
\stop

\startbrissot
I'm very sorry, my brothers. It isn't the right time.
\stop

\stopcomponent
