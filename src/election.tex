\startcomponent election

\scene

\open Fall 1791. The H\^otel des Monnaies salon. \sophie, \paine, \charlotte,
and \cabanis are waiting on \condorcet. \sophie is carrying an infant. A small
cake has been prepared.

\startsophie
Do you hear him, \eliza? Is that your father entering?
\stop

\stage Enter \condorcet.

\stage Encouraging applause.

\startcrowd
Congratulations!
\stop

\startcondorcet
Has it happened!?
\stop

\startsophie
It has! Look \eliza! Your father is now a legislator!
\stop

\startcabanis
By a meager four votes. Congratulations, my friend.
\stop

\startcondorcet
Oh, I expected as much. The constitution has cursed us with the burden of
plurality voting.
\stop

\stage \paine laughs.

\stage \paine shakes \condorcet's hand and respectfully nods.

\stage \charlotte gives \condorcet a hug.

\startcharlotte
We are so delighted to hear of your success!

Come! Come! My sister has brought us a cake!
\stop

\stage \condorcet awkwardly obliges.

\startcondorcet
Oh, oh, she has?
\stop

\startcharlotte
\(to \sophie)
So very shy for a legislator.
\stop

\startcondorcet
You all shouldn't have gone through the trouble for me.
\stop

\stage \cabanis laughs.

\startsophie
Trouble? Oh, no. No, you misunderstand.

Our celebration is not for you.
\stop

\stage \condorcet is confused.

\startsophie[continued]
Celebration is made for the celebrators, not for the celebrated.

We join together to rejoice not in the love that is given, but in the love that
is shared.
\stop

\stage \condorcet tears up.

\startsophie[continued]
Now, eat the cake as a celebrator - it was expensive.
\stop

\stage \condorcet nods and begins to partake.

\startcabanis
So what's this about - eh, what - plurality voting?
\stop

\stage \charlotte slaps his arm.

\startcharlotte
\cabanis! We're here to celebrate. It's no place to bring up politics.
\stop

\startcabanis
Dear, we are literally celebrating the man getting elected as a legislator.

Politics is the very reason we're here!
\stop

\startcharlotte
You know what I mean.
\stop

\startcabanis
That wasn't even my question!
\stop

\startsophie
Enough.
\stop

\startcondorcet
No need to worry, \charlotte.

Indeed, it's actually a wonderful question! I'd be delighted to speak on it.
\stop

\startcabanis
See?
\stop

\startsophie
Enough!
\stop

\startcondorcet
In the science of decision making, plurality voting is a form of a election
where preferences are aggregated exclusively by the top preference of each
participant.

One ballot, one vote.
\stop

\startpaine
Mm.
\stop

\startcondorcet
Yep, plurality voting is the burden of the United States as well.
\stop

\startcabanis
What's so wrong with it?
\stop

\startcondorcet
Plurality elections incentivise voters to lie about their preferences in order
to gain a strategic advantage in the results.

Consider that in a single-winner system, where each voter is given only one
vote, there is no incentive to vote for any candidate that is unlikely to win.

In other words, the voters do not submit their ballot based on what they truly
prefer, but rather based on what they believe other voters will prefer.

This is an example of a voting method that is not strategy-proof - and its use
is simply unacceptable.
\stop

\startcabanis
What alternative method would be strategy-proof?
\stop

\startcondorcet
I don't know yet.
\stop

\startcabanis
What do you mean?
\stop

\startcondorcet
I've spent years trying to figure out a proper method for voting that does not
incentivize strategic voting.

So far, everything I have come up with has some kind of flaw or exploit that
can't be avoided.
\stop

\startcharlotte
I don't understand.
\stop

\startcondorcet
Oh, please! I love questions!
\stop

\startcharlotte
You use the word "strategy" like it's a bad thing, but isn't it a good thing to
vote with your peers in mind?
\stop

\startcondorcet
An insightful perspective! Another way to see the issue with strategy is to
think about it as causing unexpected or irrational outcomes.

In plenty of cases, this can often be thought of as irrelevant alternatives on
the ballot having a surprising effect on the final decision.

Allow me to demonstrate with a story.
\stop

\startcharlotte
Oo! A story! You were right, \cabanis; this was an excellent choice of topic!
\stop

\startcondorcet
So, suppose it is after dinner at the palace of Versailles, and Marie
Antionette is ordering desert.
\stop

\startcharlotte
Ordering cake, no doubt!
\stop

\startcondorcet
Ordering cake, indeed! A servant approaches the queen and informs her that they
can prepare either vanilla cake or chocolate cake.

The queen says to the servant, "I will have the vanilla cake."

The servant bows and proceeds to the royal kitchen.

But after a moment, the servant returns and informs the queen that carrot cake
is also available.

\(beat)
"Ah," replies the queen, "in that case, I shall have chocolate!"
\stop

\stage All laugh together.

\startcharlotte
I suppose the queen is up to some secret scheme involving her choice of cake!
\stop

\startcondorcet
Precisely!
\stop

\stopcomponent
