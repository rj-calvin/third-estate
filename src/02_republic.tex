\startcomponent 02_republic

\scene

\open \condorcet has begun ignoring the discussion.

\stage Enter \voltaire.

\startvoltaire
Republic, republic! What even is a republic?
\stop

\stage Enter \leibniz.

\startleibniz
Ah! The answer is simple! A republic is a category of monad!
\stop

\startvoltaire
Oh, \leibniz. Whatever shall we do with you?
\stop

\startleibniz
Erm, perhaps more clarity shall repair my rhetoric.
\stop

\stage \leibniz prepares a blackboard:

\startdrawing
\startMPcode
\stopMPcode
\stopdrawing

\startleibniz[continued]
You see, a republic is a particular kind of monad that is equipped with a
decision function that aggregates the preferences of a collection of
individuals, members, deputies, jurors, citizens, what have you!

This way, the qualities that one observes of this monad represent a consensus
derived from the given collection and is contingent on the choice of decision
function.

These qualities thus allows the many to speak as one!

How exciting!
\stop

\startvoltaire
You truly believe that this --?
\stop

\startleibniz
-- Ah! But now consider, dear \voltaire! That if we take the monad of a
republic, then compose this with a writer monad, one could thus create a system
that records the successive decisions produced by our assembly.

Taken as a whole, this would form a ledger that could, in theory, track the
errors and progress of the republic based on the diverse preferences and
experiences of its members!
\stop

\startvoltaire
Ha! In theory, of course!

To those who maintain their sanity, a republic may be better described as a
government operated by elected representatives.
\stop

\startleibniz
But the election of representatives is simply a particular choice of decision
function! Observe:
\stop

\startdrawing
\startMPcode
\stopMPcode
\stopdrawing

\startleibniz[continued]
We can conceptualize the election of representatives as a partition of our
participants such that a decision function allows for members of one group to
rank their preferences of the members belonging to the other.

This other group is our assembly of representatives who use a second decision
function to aggregate their preferences based on the ranking given to them by
their constituency.
\stop

\startvoltaire
And if all the representatives were to die without warning? What should be done
then?
\stop

\startleibniz
You misunderstand. I speak about an abstraction.
\stop

\startvoltaire
Correct. Now answer me.
\stop

\startleibniz
Well... I'm not sure. I haven't considered.
\stop

\stopcomponent
