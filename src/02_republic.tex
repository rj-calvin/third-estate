\startcomponent 02_republic

\scene

\open \condorcet has begun ignoring the ongoing discussion.

\stage Enter \voltaire.

\startvoltaire
Republic, republic! What even is a republic?
\stop

\stage Enter \leibniz.

\startleibniz
Ah! The answer is simple! A republic is a category of monad!
\stop

\startvoltaire
Oh, professor \leibniz. Whatever shall we do with you?
\stop

\startleibniz
Erm, perhaps more clarity shall repair my rhetoric.
\stop

\goodbreak
\stage \leibniz prepares a blackboard:

\startdrawing
\startMPcode
draw fullcircle scaled 2cm;
\stopMPcode
\stopdrawing

\startleibniz[continued]
You see, a republic is a particular kind of monad that is equipped with a
decision function that aggregates the preferences of a collection of
individuals, members, deputies, jurors, citizens, what have you!

This way, the qualities that one observes of this monad represent a consensus
derived from the given collection and is contingent on the choice of decision
function.

These qualities thus allows the many to speak as one!

How exciting!
\stop

\startvoltaire
And yet, what exactly --
\stop

\startleibniz
-- Ah! But now consider, dear \voltaire! That if we take the monad of a
republic, then compose this with a writer monad, one could thus create a system
that records the successive decisions produced by our assembly.

Taken as a whole, this would form a ledger that could, in theory, track the
errors and progress of the republic based on the diverse preferences and
experiences of its members!
\stop

\startvoltaire
Ha! In theory, of course!

To those who maintain their sanity, a republic may be better described as a
government operated by elected representatives.
\stop

\startleibniz
But the election of representatives is simply a particular choice of decision
function! Observe:
\stop

\goodbreak
\stage The blackboard presents:

\startdrawing
\startMPcode
draw fullcircle scaled 2cm;
\stopMPcode
\stopdrawing

\startleibniz[continued]
We can conceptualize the election of representatives as a decision function
composed of two parts.

The first selects a partition of our assembly and allows members of one group
to rank their preferences for members belonging to the other.

The second part allows these selected representatives to then aggregate their
preferences based on the ranking given to them by their constituency.
\stop

\startvoltaire
And if all the representatives were to die without warning? What should be done
then?
\stop

\startleibniz
You misunderstand. I speak about an abstraction.
\stop

\startvoltaire
Correct. Now answer me.
\stop

\startleibniz
I'm not sure. I haven't yet considered.
\stop

\startvoltaire
I repeat, "whatever shall we do with you?"
\stop

\startleibniz
I fail to see how ignorance implies incompetence.
\stop

\startvoltaire
You fail to see many things.

Your incessant calculations keep you blind.
\stop

\stopcomponent
