\startcomponent 02_republic

\scene

\open \condorcet has begun ignoring the ongoing discussion.

\stage Enter \voltaire.

\startvoltaire
Republic, republic! What even is a republic?
\stop

\stage Enter \leibniz.

\startleibniz
Ah! The answer is simple! A republic is a category of monad!
\stop

\startvoltaire
Oh, professor \leibniz. Whatever shall we do with you?
\stop

\startleibniz
Erm, perhaps more clarity shall repair my rhetoric.
\stop

\goodbreak
\stage \leibniz prepares a blackboard:

\startdrawing
\startMPcode
draw fullcircle scaled 2cm;
\stopMPcode
\stopdrawing

\startleibniz[continued]
You see, dear \voltaire, a republic is a particular kind of monad that is
equipped with a decision function that aggregates the preferences of a
collection of members, deputies, jurors, citizens, what have you!

This way, the qualities that one observes of this monad represent a consensus
derived from the given collection and is contingent on the choice of decision
function.

These qualities thus allows the many to speak as one!

How exciting!
\stop

\startvoltaire
To those who maintain their sanity, a republic may be better described as a
government operated by elected representatives.
\stop

\startleibniz
But the election of representatives is simply a particular choice of decision
function! Observe:
\stop

\goodbreak
\stage The blackboard presents:

\startdrawing
\startMPcode
draw fullcircle scaled 2cm;
\stopMPcode
\stopdrawing

\startleibniz[continued]
We can conceptualize the election of representatives as a decision function
composed of two parts.

The first selects a partition of our assembly and allows members of one group
to rank their preferences for members belonging to the other.

The second part allows these selected representatives to then aggregate their
preferences based on the ranking given to them by their constituency.
\stop

\startvoltaire
And if all the representatives were to die without warning? What should be done
then?
\stop

\startleibniz
You misunderstand. I speak about an abstraction.
\stop

\startvoltaire
Correct. Now answer me.
\stop

\startleibniz
I'm not sure. I haven't yet considered.
\stop

\startvoltaire
I repeat, "whatever shall we do with you?"
\stop

\startleibniz
I fail to see how ignorance implies incompetence.
\stop

\startvoltaire
You fail to see many things.

Your incessant calculations keep you blind.

\(beat)
You see, professor, I see into the future. Behold!

% CONSIDER: imagery of the street.

I feel now the brisk winter cold filter through my fabrics as I approach the
new church of humanity.

A dilapidated construction that slumps to one side, the building I see is a
relic of the Ancien R\'egime.

And now, just as then, it is the object worshipped by fools who fear the
responsibilities of god.

\(beat)
If I were a good man, the sight of this warped lumber would compel me to
suicide, but I am not a good man - I am merely man.

I know that housed within this structure are the seats that have fractured and
divided absolute power.

Thus, despite my shivering, I am compelled to gawk at this image of the whole -
this image of the spoils that one man is to gain from the lottery.

\(beat)
And it is here I stand, basking in this divine imagination, where my body is
warmed by the blood that rushes to my member.

Here I stand, member in hand, members passing by, where I cry, "republic,
republic! I, the people, worship you!"
\stop

\stage \condorcet rises from his seat to address the apparitions.

\startcondorcet
Lottery? What's this about lottery? There are no winners in a republic, nor is
the conduct of our reason a game.
\stop

\startvoltaire
Oh, child, do not be so na\"ive. It's embarrassing.
\stop

\startleibniz
The accusation of arbitrariness is indeed counterproductive, \voltaire.
\stop

\startvoltaire
Do not be swindled by this post-neo-meta-theo-physico-logician.

He will dazzle you with ideals and perfections, of geometries and algebras, but
this is slight-of-hand - a trick to draw the eye away from the ugly history
that would prove him wrong.

Because unlike his pretty circles and polygons, all forms of governance exist
based on conflict, not symmetry.

And even our learn\`ed professor will tell you that anything that leads from
contradiction is utterly arbitrary.
\stop

\stopcomponent
