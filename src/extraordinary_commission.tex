\startcomponent extraordinary_commission

\scene

% \startlacuee
% The constitution is clear. By chapter two, section one, article two, the person
% of the king is inviolable and sacred.
%
% Therefore, he cannot be tried for any crime other than in the cases listed in
% articles five through seven.
% \stop
%
% \startbrissot
% By article six, damn you! He is leading the Austrians straight to Paris!
% \stop
%
% \startlacuee
% That would be the Duke of Brunswick.
%
% \louis does not lead anyone but the French, as per article two.
% \stop
%
% \startsplit
%
% \startguadet
% That's ludicrous! The emigrants fight in his name, thus this clearly falls
% within the principles of article six.
% \stop
%
% \split
%
% \startroyalist[number=1]
% Do you not read the word sacred!? Is nothing sacred to you, \brissot?
% \stop
%
% \stopsplit
%
% \startroyalist[number=2]
% We are not at war with the emigrants. We are at war with Austria. Last I
% checked, the Austrians fight in the name of their {\emph own} king.
% \stop
%
% \startguadet
% But it's the {\emph principle} of the article! Collusion with the enemy is the
% greatest of possible treasons!
% \stop
%
% \startroyalist[number=2]
% What proof do you have of collusion!?
% \stop
%
% \startvergniaud
% Proof? Oh, where were you in the Roman senate when Cicero denounced the
% conspiracy of Catiline?
%
% Would you have asked for proof then?
% \stop
%
% \startlacuee
% Gentlemen, evidence or no, collusion is not covered by the terms of the
% constitution.
%
% Since the Austrians are not led by \louis, nor do they fight in his name, the
% king, therefore, cannot be charged for having violated article six.
% \stop
%
% \startthuriot
% To hell with the constitution!
%
% No document can foresee every possible weapon a tyrant can wield against his
% own country!
%
% \beat
%
% We have declared to the people that the fatherland is in danger.
%
% We formed this committee with the promise to solve the crisis.
%
% Yet, now we allow this document to paralyze us?
%
% If this constitution should stand in our way from protecting the nation, then I
% say we burn it!
% \stop
%
% \startsplit
%
% \startroyalist[number=3]
% You should be hanged for that kind of insolence!
% \stop
%
% \split
%
% \startlacuee
% No committee is above the constitution!
% \stop
%
% \stopsplit
%
% \startvergniaud
% \thuriot, calling for extremes will only doom our cause.
% \stop
%
% \startroyalist[number=4]
% {\emph Our} cause? I have no cause with you! This is factionalism!
% \stop
%
% \startisnard
% Says the whoreson who voted to pardon \lafayette for having attempted~--!
% \stop
%
% \startcondorcet
% -- Enough! Enough!
%
% \beat
%
% While spoken rather uncharitably... I confess that there is some truth to the
% spirit of Monsieur \thuriot's objections.
% \stop
%
% \startlacuee
% What!?
% \stop
%
% \startcondorcet
% The constitution grants the people of France with the imprescriptible right of
% their sovereignty.
%
% All powers of our government are emanated from this sovereignty.
%
% Is this not true, Monsieur \lacuee?
% \stop
%
% \stage Pause.
%
% \startcondorcet
% The public is becoming increasingly desperate, gentlemen.
%
% There is talk that the nation manifest their will through the vote of a new
% National Convention, formed of representatives invested with unlimited powers.
%
% If the people are not convinced of their safety soon~- if threats are not
% addressed~- it is very possible we shall indeed see our constitution rendered
% to ashes.
% \stop
%
% \startroyalist[number=5]
% Monsieur \condorcet. Might I object?
% \stop
%
% \startcondorcet
% Certainly. What say you, monsieur?
% \stop
%
% \startroyalist[number=5]
% The call for a National Convention is clearly a call for anarchy.
%
% By what audacity do you argue for its virtues?
% \stop
%
% \stage \condorcet chuckles.
%
% \startcondorcet
% That was the most polite insult I've heard all day, monsieur.
%
% I feel compelled to thank you for it!
%
% \beat
%
% But no. I do not suggest that we simply stand back and allow this, I am simply
% making a point.
%
% Namely, that if the people of France deem its representatives incapable of
% acting in its interests, this self-same plurality can, at any time, choose to
% take matters into their own hands.
%
% \beat
%
% Therefore, to avoid insurrection, we must prove to the nation that their safety
% is assured.
%
% We must ensure that we are not paralyzed by the imaginary divisions between
% ourselves.
%
% We must act to protect the stability of our laws and maintain the trust of the
% nation who has trusted us with their representation.
% \stop
%
% \stage The \brissotins applaud.
%
% \startcondorcet
% Gentlemen, I encourage you to reduce your enthusiasm.
%
% To eliminate our prejudices and divisions, it is imperative that we begin by
% relinquishing our attachment to sentiments and submit ourselves only to the
% incontrovertible power of reason.
% \stop
%
% \startbrissot
% What do you propose, then?
%
% Those citizens who call for a National Convention are the same that call for
% the forfeiture of the king.
%
% Anything less, and we will have the insurrection they have sworn themselves to.
% \stop
%
% \startcondorcet
% The people only mean to bargain so that their fears are treated with the
% severity they believe is deserved.
%
% But equivalent outcomes can be achieved by lesser measures.
%
% Rest assured that a well-tempered address delivered with the confidence of
% reason to the point of obviousness will surely calm their spirits.
%
% \beat
%
% In our shared interest of preventing an insurrection, we are compelled to act
% boldly, but must do so within the restrictions of our constitution.
%
% And so, since the constitution does not permit the king's forfeiture, I would
% welcome any alternative seeking a compromise bold enough to unite our votes!
% \stop
%
% \stage \gensonne applauds, then stands.
%
% \startgensonne
% I agree with Monsieur \condorcet.
%
% I motion that the commission petition the assembly to vote on the king's
% suspension under the terms of chapter two, section one, article seven.
% \stop
%
% \startlacuee
% The condition of the king's flight from the country!? By what~--?
% \stop
%
% \stage The \brissotins applaud.
%
% \startcondorcet
% Very well!
%
% By show of hands, we will vote on the motion presented by Monsieur \gensonne.
% \stop
%
% \stage Ten to the left raise their hands. Ten to the right raise theirs.
%
% \stage \thuriot looks left, then right, then raises his hand.
%
% \startcondorcet
% The motion passes with a majority of one.
% \stop
%
% \stage The \brissotins applaud.

\stopcomponent
