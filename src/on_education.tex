\startcomponent on_education

\scene

\open The Salle du Man\`ege. \gensonne presides over the \assembly~- who are
all engaged in discussion amongst themselves.

\stage \gensonne stands and rings a bell.

\startgensonne
Messieurs! Silence!
\stop

\stage The \assembly lowers the volume of their discourse.

\startgensonne
Silence!
\stop

\stage No change. \gensonne accepts defeat.

\startgensonne
MESSIEURS!

I understand that you are all anxious for the arrival of the king.

But be patient! I am told he has been delayed.
\stop

\stage Grumbles from the \assembly.

\startgensonne
But that is no matter, for our docket provides us the means to tide us over
with a special report from our esteemed scholar: Monsieur \condorcet.

\beat

On behalf of the committee of public instruction, Monsieur \condorcet presents
us the committee's proposal for establishing France's institutions of free and
universal education for the nation.

No longer shall our children suffer the indignation of corrupt priesthoods and
nunneries!
\stop

\stage Applause from the \leftminority. The \rightminority sneer and whisper to
their neighbors. The rest continue their chatting.

\startgensonne
\condorcet, the floor is yours.
\stop

\stage Light applause.

\stage \condorcet stands, carrying a stack of papers in his arms. He
distributes these documents throughout the \assembly.

\stage \condorcet[thephilosopher] ascends the tribune.

\startcondorcet
G-good morning, messieurs.

\beat

It is to offer all individuals of the human species the means to provide for
their needs; to ensure their independent prosperity; to know and exercise their
own rights; and thereby guarantee the universal equality necessary for the
progress of enlightenment; that is to be the essential purpose of public
education.

\beat

It has been the great pleasure of the committee of public instruction, the
noble coalition that I am proud to speak on the behalf of, to have dedicated
these last months to the plan for France to institute schooling not only for
our children, but for any man or woman who seeks to break the... the {\emph
		shackles} of ignorance!
\stop

\stage Pause. Idle chat.

\startcondorcet
It has been in this persuasion that the documents you hold in your hands have
been printed.

Contained in the pamphlets I have distributed is the general structure and
organization of the institutions necessary for a sound system for teaching the
arts and sciences to all citizens of France.
\stop

\stage Pause. Idle chat.

\startcondorcet
Friends, there are a great many ways that our revolution may one day end in
failure, and the essential cause for all of them will be found in a failure of
education.

All of these failures will manifest in the citizen who can say: "the law has
assured me complete equality of rights, but has denied me the means to
understand them."

Education, therefore, must be made as universal as the rights it ensures.

\beat

To this end, we have decided that schools are to be distributed uniformly to
all neighborhoods of four hundred households where weekly lectures are to be
made open to all citizens who can attend them.

We make explicit that no person is to be denied access to these resources based
on the circumstances of their age, their sex, or their race;

and that all teachers are to be compensated through public funds rather than by
entrance fees or private scholarships that would discriminate access based on
wealth.

\beat

This right of citizens to have access to education at all stages of life, and
free of charge, is made even more essential as the arts continue to be
perfected.

As the machines used in the production of goods are improved through
innovations and inventions, the labor of the artisan will naturally tend toward
simpler, more repetitive motions.

Eventually, these occupations will cease to engage the mind entirely, and
therefore, the perfection of manufacturing will result in discontent for a
substantial part of the human race.

Thus, if more extensive education is not provided to this same class of
citizens, there will be no respite from the monotony of their labor, and thus,
no opportunities for them to grow beyond the simplest means of survival.

This will inevitably result in a humiliating inequality that would sow the
seeds of dangerous unrest.
\stop

\stage A messenger arrives and whispers into the ear of \gensonne.

\startcondorcet
Finally, we consider the independence of education from political and religious
influence as an essential right of~--
\stop

\startgensonne
-- My apologies, monsieur! My apologies, messieurs!
\stop

\stage The \assembly silences their chatter. \condorcet looks bewildered.

\startgensonne
I have been informed that the King of the French has arrived.
\stop

\stage Applause and cheer.

\stage Enter \louis\ - who is dressed extravagantly.

\stage \louis[xvi] ascends the tribune.

\stage \condorcet has failed to move - he hastens to rectify this.

\startcondorcet
Votre majesté!
\stop

\startanonymous
By law, he is to be addressed as the King of the French!
\stop

\stage An uproar amongst the \assembly.

\stage \louis ignores the commotion, pulls out a parchment and attempts to read
from it~- then realizes he can't read it without his spectacles.

\stage \gensonne rings a bell.

\startgensonne
DECORUM!
\stop

\stage A servant of the court procures a pair of spectacles. \louis dons them.

\stage Pause. Silence.

\startlouis
By the terms of the constitution that bestow me the right to execute the law, I
formally propose to the Legislative Assembly the declaration of war against the
King of Hungary and Bohemia.
\stop

\stage The \assembly bursts into cheer!

\startassembly
Long live the king! Long live the nation! Long live the law!
\stop

\stage \louis dismounts the tribune and sits in a chair identical to the one
\gensonne is seated in.

\startrightminority
Motion to adjourn! Motion to adjourn!
\stop

\startleftminority
A call for delay is a call for death!
\stop

\startgensonne
Decorum!

\beat

Motion to adjourn denied.

I bring the motion of our King to declare war on the kingdom of Bohemia and
Hungary to a vote by nominal call.
\stop

\stage Great applause from the \assembly!

\direction Lights down on all but \condorcet.

\stage Enter \leibniz.

\startleibniz
The right of humanity that education be independent of political and religious
influence is derived from two observations.
\stop

\startsplit

\startgensonne
Albitte!
\stop

\split
\leavevmode

\startanonymous[cue={Albitte}]
Oui!
\stop

\stopsplit

\stage Applause.

\startleibniz
First is the observation that all people are gifted a perfectibility whose
limits, if any, exist far beyond what we can conceive.
\stop

\startsplit

\startgensonne
Audrein!
\stop

\split
\leavevmode

\startanonymous[cue={Audrein}]
Oui!
\stop

\stopsplit

\stage Applause.

\startleibniz
And the second is the observation that all exercise of this faculty is
contingent on the revelation and proliferation of new truths.
\stop

\startsplit

\startgensonne
Basire!
\stop

\split
\leavevmode

\startanonymous[cue={Basire}]
Oui!
\stop

\stopsplit

\stage Applause.

\startleibniz
Therefore, the total independence of opinion in the domain of education is
essential for forbidding any authority from establishing a doctrine that
prevents the teaching of truths contrary to superstitious dogma or momentary
agendas.
\stop

\startsplit

\startgensonne
\brissot!
\stop

\split
\leavevmode

\startbrissot
OUI, MESSIEURS! OUI! TO ARMS!
\stop

\stopsplit

\startleftminority
TO ARMS! TO ARMS!
\stop

\stage Applause.

\stage Enter \voltaire.

\startvoltaire
Citizens, patriots, good and gentle friends of the nation!

Lend me your ears to my pleas for our nation's future!

The plan we present to the assembly has been the culmination of what several
centuries of observation has taught us about the progress of the human mind.
\stop

\startsplit

\startgensonne
Carnot!
\stop

\split
\leavevmode

\startanonymous[cue={Carnot}]
Oui!
\stop

\stopsplit

\stage Applause.

\startvoltaire
I warn you, messieurs, that without education, without the hope it gives us for
ensuring the prosperity of our future, all such progress will cease.

Our revolution would enter into a constant state of regression until the day it
quietly slips back into the chains of feudalism.
\stop

\startsplit

\startgensonne
Chabot!
\stop

\split
\leavevmode

\startanonymous[cue={Chabot}]
Oui! To arms!
\stop

\stopsplit

\startleftminority
Hear, hear! Long live the nation!
\stop

\stage Applause.

\startvoltaire
It is our hope that the plans set by the committee of public instruction have
succeeded in hastening the day when all prejudices will lose their influence on
society.

When all errors will have been forgotten.
\stop

\startleibniz
And when the incremental improvement of our laws can be assured without
confusion or controversy.
\stop

\startgensonne
\condorcet!
\stop

\startleibniz
But that day yet hides beyond the limits of our lifetimes - far into our
distant future.
\stop

\startvoltaire
Thus, our objective today, messieurs, is to accelerate its arrival, to build
our institutions to foster the enlightenment of our friends and neighbors, and
to remind us to always remain vigilant for the happy day~--
\stop

\startgensonne
-- \condorcet!
\stop

\stage \condorcet suddenly finds himself on the tribune.

\startleibniz
For the happy day when the institutions we create today may finally be rendered
useless!
\stop

\stage Exit \voltaire and \leibniz.

\stage Pause.

\stage The \assembly begins to whisper.

\startanonymous
Et toi...? Et toi!?
\stop

\startgensonne
Et toi, monsieur?
\stop

\stage \condorcet begins to pant.

\startassembly
ET TOI!? ET TOI!? ET TOI!?
\stop

\direction \thestranger whispers to \condorcet.

\startthestranger
Et toi, \brissot?
\stop

\startcondorcet
It... it is by my hatred for war... that I choose to declare it.
\stop

\stage The \assembly exchange confused glances.

\startbrissot
LONG LIVE THE PHILOSOPHER!
\stop

\stage Polite applause.

\stopcomponent
