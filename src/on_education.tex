\startcomponent on_education

\scene

\open The Salle du Man\`ege. \gensonne presides over the \assembly - who are
all engaged in discussion amongst themselves.

\stage \gensonne stands and rings a bell.

\startgensonne
Gentlemen! Silence!
\stop

\stage The \assembly lowers the volume of their discourse.

\startgensonne
Silence!
\stop

\stage No change. \gensonne accepts defeat.

\startgensonne
Today, gentlemen! Today, our docket begins with a report from the committee of
public instruction.

It is my understanding that the committee has concluded their work on a
proposal for the standards by which the legacy of enlightenment shall pass to
the future generations of France free from the yoke of corrupt priesthoods and
nunneries!
\stop

\stage Applause from the \rightminority. The \leftminority sneer and whisper to
their neighbors. The rest continue their chatting.

\startgensonne
The report is to be delivered by our esteemed scholar: Monsieur \condorcet.
\stop

\stage Applause.

\stage \condorcet stands, carrying a stack of papers in his arms. He
distributes these documents throughout the \assembly.

\stage \condorcet[thephilosopher] ascends the tribune.

\startcondorcet
G-gentlemen...

\beat

It is to offer all individuals of the human species the means to provide for
their needs; to ensure their independent prosperity; to know and exercise their
own rights; and thereby guarantee the universal equality necessary for the
progress of enlightenment; that is to be the essential purpose of public
education.

\beat

It has been the great pleasure of the committee of public instruction, the
noble coalition that I am proud to speak on the behalf of, to have dedicated
these last months to the plan for France to institute schooling not only for
our children, but for any man or woman who seek to break the shackles of
ignorance.

It has been in this persuasion that the documents you hold in your hands have
been printed.

Contained in these pamphlets is the general structure and organization of the
institutions necessary for a sound system for teaching the arts and sciences to
the citizens of France.
\stop

\stage Pause. Idle chat.

\startcondorcet
There are a great many ways that our revolution may one day end in failure, and
the essential cause for all of them will be found in a failure of education.

All of these failures will manifest in the citizen who can say: "the law has
assured me complete equality of rights, but has denied me the means to
understand them."

Education, therefore, must be made as universal as the rights that it exists to
guarantee.

\beat

To this end, we have ensured that schools are to be distributed uniformly to
all neighborhoods of four hundred households where weekly lectures are to be
made open to all citizens who can attend them - no matter their age.

We make explicit that no person is to be denied access to these resources based
on the circumstances of their sex or race, and that all teachers are to be
compensated through public funds rather than by entrance fees or private
scholarships.

\beat

The right of citizens to have access to education at all stages of life is made
even more essential as the arts continue to be perfected.

As the machines used in the production of goods are improved through
innovations and inventions, the labor of the artisan will naturally tend toward
simpler, more repetitive motions.

Eventually, these occupations will cease to engage the mind entirely, and
therefore, the perfection of manufacturing will result in a sort of stupidity
for a substantial part of the human race.

Thus, if more extensive education is not provided to this same class of
citizens, there will be no respite from the monotony of their labor, and thus,
no opportunities for them to grow beyond the simplest means of survival.

This will inevitably result in a humiliating inequality that would sow the
seeds of dangerous unrest.
\stop

\stage Pause. Idle chat.

\startcondorcet
Finally, we consider the independence of education from political and religious
influence as an essential right of humanity.

This follows from two observations.

First is the observation that all people are gifted a perfectibility whose
limits, if any, exist far beyond what we can conceive.

The second is the fact that knowledge of new truths is essential to the
exercise of this blessed faculty.

Thus, how can any authority dare to announce the lines that distinguish truth
from error without stifling the progress of the human mind?

The independence of opinion in education is, therefore, critical for ensuring
that no power can prevent the teaching of truths contrary to superstitious
dogmas and momentary policies.

\beat

Consider the lessons that we may learn from history.

We might first cite India and Egypt, whose ancient knowledge astonishes us
to this day, and under whom the human mind fell into the most shameful
ignorance when religious powers seized the exclusive right to instruct the
youth.

We could cite China, which produced the most exceptional insights in the arts
and sciences, and who suddenly halted all progress, for thousands of years,
by making public education the function of government.

We could also cite the decadence into which reason and genius fell from its
highest glory among the Romans and Greeks when teaching passed from the hands
of philosophers to those of priests.

Let us, therefore, be terrified of following in this path, and let us shun any
attempt to reestablish a dark age that will reverse the progress of the human
species.
\stop

\stage The \brissotins, who had been paying attention, applaud.

\startcondorcet
Gentlemen, the plan we have presented to the assembly has been derived from
observations of the current state of enlightenment in Europe.

It is, as we believe, the culmination of what several centuries of observation
has taught us about the progress of the human mind in the arts and sciences.

It is our hope that these plans may nurture the seeds that will accelerate the
arrival of a day when the existence of errors will be eliminated and when
prejudices will lose all influence in society.

But this day remains far into our future, thus our objective today is to
accelerate its arrival; and, in building our institutions for this purpose, we
must constantly remain vigilant of the happy day when they shall become
useless!
\stop

\stage The \brissotins give a standing ovation.

\stage Polite applause follows from the rest of the \assembly.

\stopcomponent
