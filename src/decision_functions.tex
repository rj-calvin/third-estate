\startcomponent decision_functions

\scene

\open \condorcet is writing at his desk.

\stage Enter \voltaire.

\startvoltaire
Republic, republic! What even is a republic?
\stop

\stage Enter \leibniz.

\startleibniz
Ah! The answer is simple! A republic is a category of monad!
\stop

\startvoltaire
Oh, god, keep it in your pants, professor.

Nobody wants to see your monads!
\stop

\startleibniz
A monad is nothing to fear, dear \voltaire!

Monads are simply the structures of our reason! Nothing but harmless
abstractions.
\stop

\startvoltaire
Like the kind that burn heretics at the stake? Or the kind that don't exist?
\stop

\startleibniz
Neither!
\stop

\startvoltaire
Oh, \leibniz, whatever shall we do with you?
\stop

\stage \condorcet begins to think.

\direction \thestranger whispers to \condorcet.

\startcondorcet
But... what should be done to convince the people of the virtues of a republic?

How do we refute the prejudices of the royalists?
\stop

\startvoltaire
Ah, of course. The age old question: without a divine monarch, how will we
ever protect ourselves from tyranny?

\beat

That's a tough one. A real stinker.
\stop

\startleibniz
Ah, well! The trouble with republics is that they depend on having a proper
decision function.

Particularly, a decision function that does not incentivize strategy.

\beat

You see, a decision function is a method of aggregating the preferences of an
assembly into a single preference representing the whole.

Allow me to illustrate!
\stop

\goodbreak
\stage \leibniz wheels in a blackboard:

\startformula[modern]
\startmathalignment[n=3]
\NC \text{First Estate} \NC :\; \NC A > B > C \NR
\NC \text{Second Estate} \NC :\; \NC B > C > A \NR
\NC \text{Third Estate} \NC :\; \NC C > B > A \NR
\stopmathalignment
\stopformula

\startleibniz
Suppose we have three estates that are tasked with deciding between three
policies: A, B, and C.
\stop

\startvoltaire
The first for the clergy, the second for the nobles, but who does the third
belong to?
\stop

\startleibniz
Now, policy A represents the status quo and is the preferred choice of the
first estate.

Policy B represents the moderate change that is preferred by the second estate.

Lastly, policy C represents a radical change and is preferred by the third
estate.
\stop

\startvoltaire
Interesting adjective, professor.
\stop

\startleibniz
An example of a decision function would be a pairwise majority.

Here, we see that two of three prefer policy B over A, two of three prefer
policy B over C, and two of three prefer policy C over A.

Thus, this decision function would aggregate these preferences into the result:
\stop

\stage \leibniz appends:

\startformula[modern]
\startMPcode
path p; p := fullcircle xscaled 3cm yscaled 1cm;
draw p;
label(btex $B > C > A$ etex, center p);
\stopMPcode
\stopformula

\startcondorcet
Yes. The reasoning is clear, but I see that the result is just a copy of the
second estate's preference.

Is it possible to choose a decision function that doesn't privilege one of the
estates?
\stop

\startleibniz
Ah... well, you now see the issue.

Suppose that the third estate could somehow infer the preferences of their
rivals.
\stop

\startvoltaire
Such as if, say, they're all wearing funny robes and chanting about some
ancient dogma.

Or if they all give each other funny titles to let everyone know that they
don't pay taxes.
\stop

\startleibniz
If this inference by the third estate is reliable, we can see that it is in
their interest to ensure their preference for policy C is not lost to the will
of a privileged estate.

Consider: why would the third estate submit their preferences honestly when
they know that doing so will end up favoring their rivals?

Observe that if the third estate were to strategically change their preference
as so:
\stop

\goodbreak
\stage \leibniz edits:

\startformula[modern]
\startmathalignment[n=3]
\NC \text{Third Estate} \NC :\; \NC C > A > B \NR
\stopmathalignment
\stopformula

\startformula[modern]
\startMPcode
path p; p := fullcircle xscaled 3cm yscaled 1cm;
draw p;
label(btex ? etex, center p);
\stopMPcode
\stopformula

\startleibniz
Now, two of three prefer A to B, two of three prefer B to C, and two of three
prefer C to A.
\stop

\startcondorcet
It forms a cycle!
\stop

\startleibniz
Indeed! By reporting their preferences strategically, rather than honestly, the
third estate can sabotage pairwise majority decisions.
\stop

\startcondorcet
Ah. This is troubling news.

\beat{tsk tsk}

So, what would be an example of a strategy-proof decision function, professor?
\stop

\startsplit

\startleibniz
I don't know of one, yet!
\stop

\split

\startvoltaire
I can think of one!
\stop

\stopsplit

\startcondorcet
You know of one, \voltaire?
\stop

\stage \voltaire smiles.

\startvoltaire
Oh, yes.

\beat

Suppose our decision function simply selects the first estate every time.

Then the function is strategy-proof since nobody needs to strategize.

The first estate does not since it knows that it's preferences are always going
to succeed.

And the second and third estates do not since nothing they can do matters!
\stop

\startleibniz
Alas... this is correct.

But that would be a trivial decision function. We are only interested in {\emph
		non}-trivial decision functions.
\stop

\startvoltaire
It seems pretty damn non-trivial to me.
\stop

\startcondorcet
Have faith! Such a non-trivial, strategy-proof decision function is necessary
for the advancement of progress!
\stop

\startleibniz
One step forward to a destination infinitely far away!
\stop

\stage Enter \sophie, knocking at the door.

\direction \leibniz and \voltaire disappear.

\startsophie
My love. Please keep this in your head. You keep waking the baby.
\stop

\startcondorcet
Oh! My apologies!
\stop

\stage Exit \sophie.

\stage \condorcet returns to his desk and stares at his writing.

\stage Pause.

\startcondorcet
Boy, that wasn't very helpful at all...
\stop

\stopcomponent
