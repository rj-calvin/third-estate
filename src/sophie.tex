\startcomponent sophie

\scene

\open \eliza's bedroom. \sophie is standing~- silent and still - over a cradle.

\stage Enter \condorcet.

\startsplit

\startcondorcet
\sophie. I, uh --
\stop

\split
\leavevmode

\startsophie
-- I know.
\stop

\stopsplit

\startsplit

\startcondorcet
I didn't think --
\stop

\split
\leavevmode

\startsophie
-- I understand.
\stop

\stopsplit

\stage Pause.

\startsophie[continued]
Why is it that I feel as if putting \eliza to bed has become a betrayal of my
principles?

\(beat)
Yesterday, this would be a moment for love and intimacy, but now...
\stop

\startcondorcet
But now you think of your trouble with \charlotte?
\stop

\startsophie
But why must that be so? What has changed from yesterday?

My feelings about \charlotte have nothing to do with my relationship with
\eliza, yet now one has contaminated the other.
\stop

\startcondorcet
That's not true.
\stop

\startsophie
But I feel it as so, and I must feel it for a reason.
\stop

\startcondorcet
Hm, that's true.
\stop

\stage Enter \voltaire.

\startvoltaire
Suffering in silence within her philosophy.
\stop

\stage Enter \leibniz.

\startleibniz
We can help her by examining the causes!
\stop

\startvoltaire
Causes can only go as far as our awareness of them.

Right now, we must observe her.

We must ask: where do feelings such as this come from?

From what manner of experiences?

And under what circumstances?
\stop

\startleibniz
I see, but this would be merely speculation.
\stop

\startvoltaire
I disagree. This is an act of translation.

\sophie communicates to us with the history of our friendship.

And it is our duty to her that we decode her message.
\stop

\startleibniz
But how can you be so sure that the message is the correct one?
\stop

\startvoltaire
I can't ever be sure. All I can know is how she responds to our response.
\stop

\startleibniz
Care to demonstrate?
\stop

\startvoltaire
Sure.

\(approaches \sophie)
I see a woman who long ago, in an age half remembered, decides one night to
hide herself away from her family.

She had found a retreat within the attic of their home to read some seditious
literature that had been forbidden by her father.

I assume she must have procured it by some maniacal scheme of hers.

Her family's traditional catholic values would certainly not have made such
material easy to come by, nor easy to keep concealed.

\(breaks away)
And, ah! Of course! As I peer closer into this scene, I see that she has
made an excellent choice in sedition:

none other than that of yours truly!

I'm honored that such a young rogue would go to such lengths to read my work.

Indeed, what a reward it must be for her to have crammed herself into the
corner of this dingy attic, to feel the cozy heat of her candlelight, to
experience the pleasure of stifling her giggles in conspiracy!

\(beat)
But alas, she laughs a little too loudly and destiny would have it that her
elder sister would discover her treachery.

The sisters enter a heated exchange, culminating in one stealing the book from
the other and setting it alight with a lit candle.

The sister then tosses the burning book into the yard through the attic window,
catching a nearby tree on fire.

\(beat)
Aghast, the sisters both rush into the yard to put out the flames.

To no avail, the two whip the flames with blankets - both too terrified of the
uproarious flames to properly smother the fire.

The bright heat signals the patriarch, and the doom of impending judgement
emerges from the house.

The elder sister rushes to the father - accusing her sibling of reading
\voltaire and having become an atheist.

Oh, such cruelty!

\(beat)
The patriarch, without reason and without appeal, swiftly delivers his
punishment to the accused.

The elder sister, now victorious and smug, indulges in her virtue by returning
into the home with her father,

leaving the stupefied young woman alone to repair her soul under a lit canopy.

\(beat)
"NO," she thinks, "FUCK \voltaire!

We DO live in the best of all possible worlds, and it is imperative that
everyone be made to understand this!"

\(approaches \sophie)
She breaks down into inconsolable tears.

The kind of which forever define the axioms - the structure - of the psyche.

And thus, so it was, there lit by the burning tree, where she vowed never to
cry again.
\stop

\direction \thestranger whispers to \condorcet. \leibniz and \voltaire
disappear.

\startcondorcet
I don't understand why \charlotte refuses to learn.

\(beat)
Wait, no... that's not quite it...

\(beat)
Why she refuses to be taught by you?
\stop

\stage Pause.

\stage \sophie maintains her composure.

\startcondorcet[continued]
Ah! I think I might know!
\stop

\stage \sophie snaps to attention: "stop."

\startcondorcet[continued]
\leibniz[full] was a man who travelled throughout the world for his pursuit of
knowledge.

And during his travel, far to the east, he was introduced to the ancient
philosophy of Confucius.

\(beat)
You see, Confucius believed that there exists particular, well-defined roles in
society and that these roles must be understood and obeyed for the greater good
of humanity.
\stop

\stage \sophie begins questioning her marriage.

\startcondorcet[continued]
He describes this process using a simple phrase: "music to unite, ritual to
divide."

\(beat)
But... you see, not everyone agreed with Confucius - or, more specifically,
they disagreed with the "attitude" Confucius had about his own philosophy.

They argued that the roles of our society are shaped by the individual through
the process of seeking one's ease with nature.

And one of these philosophers, by the name of Zhuangzi, wrote a famous text to
illustrate these principles using allegories and parables.

One of these stories goes like this:

\(beat)
There once were three emperors: the emperor of the north sea, the emperor of
the south sea, and a person by the name of Wonton who ruled the land between.

Routinely, the two emperors of the two seas would unite on the land and partake
in the polite hospitality of Wonton.

And one day, the two emperors decide to thank Wonton through an act of charity.

The north and south emperors announce to Wonton: "all people have seven holes
for seeing, hearing, eating, and breathing. Wonton alone seems to lack them.
Let's try boring some holes for him."

\(beat)
So, every day, the two emperors would bore a new hole into Wonton's head.

\(beat)
And on the seventh day, Wonton dies.
\stop

\stage \sophie struggles to maintain her composure.

\startcondorcet[continued]
I... I'm sorry... I don't know how else...
\stop

\stage \sophie smothers her defenses, taking deep breaths as she does.

\stage \condorcet does his best to comfort her.

\stopcomponent
